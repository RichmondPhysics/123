\documentclass[english,twoside]{article}

\input{../labmanual_formatting_commands} %all general latex packages, commands, and definitions now here.

\begin{document}

\Lab{4}{Investigating Specific Heat}

\renewcommand{\arraystretch}{1.5}
\setlength{\arrayrulewidth}{0.2mm}
\setlength{\tabcolsep}{10pt}

%\newlength{\myMheight}
% Create the reference text for measures
%\settoheight{\myMheight}{M}



\newcolumntype{M}[1]{>{\centering\arraybackslash}m{#1}}

\makelabheader %(Space for student name, etc., defined in master.tex or labmanual_formatting_commands.tex)

\textbf{Driving Question}

Have you ever wondered why it feels cooler on a hot summer day near a large body of water as compared to inland, away from that water, or why it feels warmer next to the water late at night?

\begin{itemize} 
	\item What does the specific heat of water versus land have to do with the differential heating and cooling of water and sand? 
	\item How different are water and sand in their rates of heating and cooling?
\end{itemize}

\textbf{Background}

The specific heat of a substance (also known as its specific heat capacity) determines how quickly the temperature of that material will rise or fall when it gains or loses heat energy. Specific heat is an intrinsic property of a substance and is dependent on its molecular structure and phase. The stronger the bonds (or intermolecular attractions) are, the higher the specific heat. The higher the specific heat, the more energy is necessary to raise the temperature of a substance and the more energy must be lost to decrease its temperature. 

Liquid water has a type of intermolecular attraction (hydrogen bonding) that causes it to have a high specific heat. The hydrogen bonds can absorb a large amount of energy before they break. They keep water molecules from moving relative to each other, resulting in lower kinetic energy for the water molecules, and thus lower heat loss.

Specific heat c refers to the amount of energy needed to raise the temperature of 1 gram of a substance 1 degree Kelvin in the same phase. This is expressed in units of joules per gram-degree Kelvin (J/g$\cdot$K). Often, specific heat is expressed using the Celsius scale (J/g$\cdot^\circ$C). The specific heat of water, 4.186 J/g$\cdot^\circ$C, is often represented as its own separate measure, the calorie. 

Liquid water's specific heat is one of the highest of any substance. Therefore, liquid water requires more heat energy to increase its temperature than almost any other substance. Likewise, liquid water must lose more energy to decrease its temperature than almost any other substance.

\textbf{Materials and Equipment}

\begin{itemize}
	\item Stainless steel temperature sensors (2)
	\item Fast-response temperature probes (2)	
	\item Small tripod base, and rod 
	\item Buret clamp (2)
	\item Beaker, glass, 500-mL	
	\item Test tube, glass, $18 \times 150$-mm (large)	
	\item Beakers (2), glass, 250-mL	
	\item Calorimeter cup (2) and lid
	\item Sand, 200 g
	\item Water, 650 mL	
	\item Stirring rod
	\item Tongs
	\item Heat lamp or 150 W incandescent lamp
	\item Hot plate
	\item Mass balance or scale
\end{itemize}

\textbf{Procedure}

\textbf{Part 1: Heating and Cooling of Water versus Sand}

\begin{enumerate}
	\item Open \textit{SPARKvue} and build a page with two graphs.
	\item Connect the two temperature probes (in analog channels A and B) to your device. Select the Stainless Steel option for your probes.
	 \item Display the temperature reading of one of the probes on the $y$-axis of the first graph. Display the temperature reading of the other probe the $y$-axis of the second graph.
	\item Put 200 g of sand into a 250-mL beaker. 
	\item Put 200 g of water into another 250-mL beaker. 
	\item Place a fast-response temperature probe in each beaker. Each temperature sensor should be approximately one inch below the surface.  
	\item Place the heat lamp directly above the beakers so that both beakers receive the same amount of energy from the lamp. 
	\item	Why is it important to heat both beakers equally?
	\answerspace{1.5in}
	\item Start data recording. 
	\item Adjust the scale of the graph to show all data. 
	\item Record data for 30 seconds then turn on the light.
	\item Continue recording for 5 minutes.
	\item How much faster do you think the temperature of the sand will increase than that of the water? How much faster will it decrease when the light is turned off? Give a specific rate comparison (such as twice as fast or twice as slow).
	Note: While the data is recording for 5 minutes, you can begin setting up Part 2 of the procedure.
	\answerspace{1.5in}
	\item Turn the light off. 
	\item Continue recording data for 5 minutes. 
	\item Stop recording data. 
	\item Name the data run ``Sand and Water''. 
	\item Save your experiment. 
	\item Complete the steps in the Data Analysis section for Part 1.
\end{enumerate}

\textbf{Part 2: Specific Heat of Sand}
\begin{enumerate}
	\item How do you think the specific heat of sand will compare with the specific heat of water? Be specific.
	\answerspace{1.0in}
	\item Fill the 500-mL beaker about 3/4 full with water. 
	\item Place the beaker on the hot plate, and turn it on to the highest setting. 
	\item Bring the water to a boil.
	\item Set up the tripod base and rod while you wait for the water to boil. Fasten a buret clamp just above the beaker. 
	\item Measure the mass of the test tube: \rule{4cm}{0.15mm}                              
	\item Fill the test tube half full with sand. 
	\item Measure the mass of the sand and the test tube: \rule{4cm}{0.15mm}                             
	\item Calculate the mass of the sand alone and write the mass in Table 2 in the Data Analysis section. 
	\item Use the buret clamp to secure the test tube in the 500-mL beaker of boiling water. Make sure the sand in the test tube is below the water level. 
	\item Connect the two stainless temperature probes. These are different from the probes used in Part 1. Select the Stainless Steel option for your probes.
	\item Use the other buret clamp to suspend one of the stainless steel temperature sensors into the middle of the test tube. Do not allow the sensor to touch the bottom or sides of the test tube.
	\item Before making the calorimeter (by nesting two disposable insulated cups), make sure the lid has a hole in it that you can slide the stainless steel temperature sensor and stirring rod through. 
	\item Measure 70.0 g of room-temperature water. 
	\item Place the two disposable insulated cups together, and add the water into the top cup. 
	\item Use the other stainless steel temperature sensor to measure the temperature of the water in the insulated cup. 
	\item Start recording data, and adjust the scale of the graph to show all data.  
	\item Record data for about 600 seconds. 
	\item Turn the hot plate off.
	\item Use the same temperature sensor you used to measure the temperature of the water. Insert the temperature sensor through the hole in the lid (the lid is not on the disposable insulated cup yet). 
	\item Insert the stirring rod through the same hole and put the lid on the cup, making sure the thermometer is in contact with the water.
	\item Start recording data, and adjust the scale of the graph to show all data. 
	\item Use tongs to remove the test tube and quickly pour the contents of the tube into the water in the calorimeter. 
	\item Immediately cover the disposable insulated cup with the lid, making sure the temperature sensor doesn't touch the side or bottom of the cup. Stir the water and sand mixture.
	\item Why did you pour the sand into the water? 
Hint: The specific heat of water is a known constant: 4.186 J/g$\cdot^\circ$C.
\answerspace{1.5in}
	\item Continue stirring the water and sand mixture. 
	\item Record data until the temperature starts to level off, and then stop recording data. (This will take about 1 minute.) 
	\item Name the data run ``Sand and Calorimeter''. 
	\item Record the initial temperature of the sand when it was added to the water in Table \ref{table_specific_heat_sand} ($T_\text{initial}$). 
	\item How did the initial temperature of the water and sand added to the insulated cup compare to the final temperature of the water-sand mixture? Where did the heat energy of the sand go when you put it into the water?
	\item What was the purpose of using an insulated cup and lid rather than simply using a beaker?
	\answerspace{1.8in}	
\end{enumerate}
\newpage
\textbf{Data Analysis}

\textbf{Part 1: Heating and Cooling of Water versus Sand}
\begin{enumerate} 
	\item Sketch your data run ``Sand and Water'' on the graph. Label the axes with units and a scale, and indicate when the light was turned on and turned off. Which curve represents the sand? Which curve represents the water?

\begin{center}
\begin{minipage}[h]{1.0\linewidth}
{\par\centering \includegraphics[width=0.80\linewidth]{blank_graph.png} \par}
\end{minipage}
\end{center}

	\item Use the graph tools to determine the temperature data points specified in Table \ref{table_water_sand}
	\item Complete the calculations for Table \ref{table_water_sand}.
\begin{center}
\begin{table}[h] 
\caption{Rates of heating and cooling of water and sand}
\begin{tabular}{ | M{1.2cm} | M{1.2cm} | M{1.5cm} | M{1.5cm} | M{1.5cm} | M{1.2cm} | M{1.5cm} | M{1.5cm} |} 
 \hline
 & $T_\text{initial}$ \newline {($^\circ$C)} & $T_\text{max light on}$ \newline {($^\circ$C)} & $\Delta T_\text{heating}$ \newline {($^\circ$C)} & Rate$_\text{heating}$ \newline {($^\circ$/s)} & $T_\text{final}$ \newline {($^\circ$C)} & $\Delta T_\text{cooling}$ \newline {($^\circ$C)} & Rate$_\text{cooling}$ \newline {($^\circ$C/s)} \\ \hline
 Water & & & & & & & \\[15pt] \hline
 Sand & & & & & & & \\[15pt] \hline
\end{tabular}
\label{table_water_sand}
\end{table}
\end{center}
\end{enumerate}
\newpage	
\textbf{Part 2: Specific Heat of Sand} 
\begin{enumerate}
	\item Sketch the graph of the third data run of Temperature versus Time for the water-sand mixture. Be sure to label the $x$-axis and $y$-axis regarding parameter and units of measurement as well as the data runs.

\begin{center}
\begin{minipage}[h]{1.0\linewidth}
{\par\centering \includegraphics[width=0.8\linewidth]{blank_graph.png} \par}
\end{minipage}
\end{center}

 	\item Recordd the final temperature of the sand-water mixture ($T_\text{final}$) in Table \ref{table_specific_heat_sand}. (The initial temperature of the sand should have been entered earlier).
	\item Complete the calculations for Table \ref{table_specific_heat_sand}. Hint: Determine the amount of heat gained by the water ($Q$) using the mass of the water ($m$), the specific heat of the water ($c$), and the change in temperature of the water ($\Delta T$.) This relationship is described by the equation $Q = mc \Delta T$. 
\end{enumerate}

\begin{center}
\begin{table}[h] 
\caption{Determining the specific heat of sand}
\begin{tabular}{| M{1.5cm} | M{1.5cm} | M{1.5cm} | M{1.5cm} | M{1.5cm} | M{1.5cm} | M{3.0cm} |} 
 \hline
 Material & Mass $m$ \newline {(g)} & $T_\text{initial}$ \newline {($^\circ$C)} & $T_\text{final}$ \newline {($^\circ$C)} & $\Delta T$ \newline {($^\circ$C)} & $Q$ \newline {(J)} & Specific heat $c$ \newline {(J/g$\cdot^\circ$C)} \\ \hline
 Water & & & & & & \\[15pt] \hline
 Sand & & & & & & \\[15pt] \hline
\end{tabular}
\label{table_specific_heat_sand}
\end{table}
\end{center}
					
\textbf{Analysis Questions}
\begin{enumerate}
	\item Calculate the ratio of the sand's rate of temperature increase to the rate of the water's temperature increase during the heating condition.
	\answerspace{1.6in}

	\item Calculate the ratio of the sand's rate of decrease to the rate of the water's temperature increase during the cooling condition.
	\answerspace{1.6in}

	\item How much faster did sand heat up and cool down compared to water? How does your prediction regarding the relative rates of heating and cooling compare with the results? Give a quantitative comparison.
	\answerspace{1.6in}
	
	\item In Part 1 of this exploration, what was the independent variable and the dependent variable, and what factors did you hold constant?
	\answerspace{1.6in}
	
	\item Compare your results for Part 2 with your prediction, using specific quantities.
	\answerspace{1.6in}
	
	\item What is the relationship between the specific heat of a substance and the rate of temperature change when the energy content of the environment around it changes?
	\answerspace{1.6in}

	\item What characteristics of water account for its high specific heat?
	\answerspace{1.6in}
	
	\item In this activity, what does $Q$ represent? Why was $Q$ the same for the water and the sand?
	\answerspace{1.6in}
	
	\item List some important sources of experimental error that might occur in this activity.
	\answerspace{1.6in}
\end{enumerate}
\newpage
\textbf{Synthesis Questions}
Use available resources to help you answer the following questions.

\begin{enumerate}
	\item	Explain how the proximity to a large body of water influences weather. Provide an example.
	\answerspace{2.0in}
	
	\item Explain how the proximity to a large body of water influences climate. Provide an example.
	\answerspace{2.0in}
	
	\item Explain how a large land mass influences weather. Provide an example.
	\answerspace{2.0in}
	
	\item Explain how a large land mass influences climate. Provide an example.
	\answerspace{2.2in}
\end{enumerate}
	
\textbf{Multiple Choice Questions}
Select the best answer(s) to each of the questions or incomplete statements below.

\begin{enumerate}
	\item To complete your calculations in Part 2, you assumed that heat energy in the heated sand was transferred to the water in the insulated cup until the temperatures of the sand and water were equal. Which fundamental principles were you relying on? Choose all that apply.

\begin{enumerate}[label=\Alph*.]
  \item The laws of motion.
  \item The first law of thermodynamics (energy conservation in an isolated system).
  \item The second law of thermodynamics (heat flows spontaneously from hot to cold until thermal equilibrium).
  \item The third law of thermodynamics (entropy approaches a constant minimum as temperature approaches 0 K).
  \item The law of gravity.
\end{enumerate}
	\item All of the following noticeably affect global air circulation EXCEPT
	\begin{enumerate}[A.]
		\item Uneven heating of the earth's surface.
		\item Rotation of the earth on its axis.
		\item The difference in specific heat of water compared to that of land.
		\item Seasonal changes in temperature and precipitation.
		\item The position of Earth's moon relative to Earth.
	\end{enumerate}
	\item Compared to a substance with a low specific heat, a substance that has a high specific heat
	\begin{enumerate}[A.]
		\item Requires more heat per gram to be added to it to cause an increase in temperature.
		\item Requires less heat per gram to be added to it to cause an increase in temperature.
		\item Requires the same amount of heat per gram to be added to it to cause an increase in temperature.
		\item Has a larger mass.
		\item None of these is true.
	\end{enumerate}
	\item Water has a high specific heat because
	\begin{enumerate}[A.]
		\item It changes from a solid to a liquid phase at a relatively high temperature.
		\item It has intermolecular hydrogen bonds.
		\item It boils at 100$^\circ$C.
		\item It freezes at 0$^\circ$C.
	\end{enumerate}
	\item The high specific heat of water compared to that of land results in
	\begin{enumerate}[A.]
		\item The small range of temperatures in the oceans compared to that on land.
		\item Coastal climates that have smaller ranges of temperature compared to those of inland areas.
		\item The ability of large fresh water bodies to stay in liquid phase when air temperatures drop below 0$^\circ$C.
		\item All of these are true.
	\end{enumerate}
\end{enumerate}

\end{document}
