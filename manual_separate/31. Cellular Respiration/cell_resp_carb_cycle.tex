\documentclass[english,twoside]{article}

\input{../labmanual_formatting_commands} %all general latex packages, commands, and definitions now here.

\begin{document}

\Lab{31}{Cellular Respiration and the Carbon Cycle}

\renewcommand{\arraystretch}{1.5}
\setlength{\arrayrulewidth}{0.4mm}
\setlength{\tabcolsep}{10pt}

\newlength{\myMheight}
% Create the reference text for measures
\settoheight{\myMheight}{M}



\newcolumntype{M}[1]{>{\centering\arraybackslash}m{#1}}

\makelabheader %(Space for student name, etc., defined in master.tex or labmanual_formatting_commands.tex)

\textbf{Driving Question}

When dormant seeds are exposed to water, they begin to germinate. By measuring the level of CO$_2$ gas in closed systems containing dormant or germinating seeds, you can explore the following:
\begin{itemize}
	\item What happens to carbon dioxide gas concentrations in a closed system containing germinating seeds and air?
	\item What does cellular respiration have to do with the carbon cycle?
	\item Can germinating seeds serve as a model for cellular respiration? Why or why not?
\end{itemize}

\textbf{Background}

In the germinating seed, cells within the growing plant embryo use energy-storage molecules (oils, starches, sugars) stored in the seed for food to fuel its life processes. The energy from the food is extracted through a process of cellular respiration. In a series of hundreds of biochemical reactions, the carbon atoms that are bound in the large energy-storage molecules are combined with oxygen gas to produce carbon dioxide gas, water vapor, and energy that is vital to living processes.

The carbon dioxide gas that is released into the environment is then available to plants for use in the process of photosynthesis. During photosynthesis, plants trap energy from the sun to convert carbon dioxide gas and water into complex sugar molecules that store potential chemical energy. The carbon dioxide gas and water vapor molecules released into the atmosphere during cellular respiration also function as natural ``greenhouse gases'', causing the atmosphere to retain more of the sun's energy through the natural greenhouse effect.

\textbf{Materials and Equipment}
\begin{itemize}
	\item Data collection system
	\item Knife or scalpel 
	\item CO$_2$ gas sensor
	\item Dry bean seeds (11)
	\item Germinating bean seeds (11)
	\item Sampling bottle (included with sensor)	
	\item Dissecting microscope or magnifying glass	
\end{itemize}

\textbf{Safety}

Use care with the knife or scalpel. 

\textbf{Procedure}

\textit{Part 1: Observing and comparing the morphology of dormant and germinating bean seeds}
\begin{enumerate}
	\item Obtain one dry seed and one soaked seed. Use a knife or scalpel to bisect the seeds longitudinally into halves.
	\item Use a magnifying glass or dissecting microscope to observe the interior of the seed.
	\item Sketch the major morphologic features of the bean seed. Label the cotyledon, embryo, and seed coat.
	\answerspace{4.0in}
	\item List some differences you observe in the appearance of the dry versus the soaked seed.
	\answerspace{2.5in}
	\item What is a sign that the seed has changed from a dormant state to a growing state, in other words, that the seed is germinating?
	\answerspace{2.5in}
\end{enumerate} 

\textit{Part 2: Comparing the CO$_2$ gas concentrations of a closed system containing dormant seeds versus a closed system containing germinating seeds}
\begin{enumerate}[start=6]
	\item Open \textit{SPARKvue} and build a page with one graph. 
	\item Connect the CO$_2$ gas sensor.
	\item Display CO$_2$ gas concentration on the $y$-axis of the graph.
	\item Why is CO$_2$ plotted as a function of time? What is another way you could represent or summarize this data?
	\answerspace{1.5in}
	\item Put 10 dry seeds into one sampling bottle and 10 soaked seeds in the other sampling bottle.
	\item Insert the end of the CO$_2$ gas sensor into the sampling bottle containing the dry seeds and firmly plug the end of the sampling bottle with the rubber stopper.
	\item Why are you using dry and soaked seeds?
	\answerspace{1.5in}
	\item Predict what will happen to the CO$_2$ concentration during data recording? Why?
	\answerspace{2.0in}
	\item Start data recording. Note: Avoid bumping the equipment because jarring or bumping the sensor might cause the sensor to record erratically.
	\item Data collection will stop automatically after 10 minutes. Write the run number here:             
	\item Switch the sensor to the other sampling bottle.
	\item Start recording data.
	\item Data collection will stop automatically after 10 minutes. Write the run number here:
	\item Save your experiment, and clean up according to your teacher's instructions.
\end{enumerate}
\newpage
\textbf{Data Analysis}
\begin{enumerate}
	\item Display both data runs on the graph display. 
	\item Adjust the scale of the graph to show all data.
	\item On the graph below, sketch the plotted data on your graph display. Be sure to label the $x$-axis and $y$-axis regarding parameter and units of measurement. Label each data run.

{\par\centering \includegraphics[width=0.70\linewidth]{blank_graph.png} \par}
 
	\item Find the initial and final CO$_2$ concentrations for each run and record them in the data table.
	\item Calculate the change in CO$_2$ concentration for each experimental condition and record these values in Table 1.

\begin{center}
\begin{tabular}{ | M{2.5cm} | M{3.5cm} | M{3.5cm} | M{3.5cm} |}
 \hline
	\multicolumn{4}{|c|}{Table 1: CO$_2$ concentration of dry seeds and germinating seeds} \\ \hline
 & Initial CO$_2$ Concentration \newline (ppm) & Final CO$_2$ Concentration \newline (ppm) & Change in CO$_2$ Concentration \newline (ppm) \\ \hline
 Dry seeds & & & \\[15pt] \hline
 Germinating seeds & & & \\[15pt] \hline
\end{tabular}
\end{center}
\end{enumerate}
		
\textbf{Analysis Questions}
\begin{enumerate}
	\item Compare the change in CO$_2$ concentration for dry seeds versus soaked seeds.
	\answerspace{1.5in}
	
	\item What can you conclude about the soaked seeds regarding CO$_2$?
	\answerspace{1.5in}
	
	\item Compare your prediction to the data you collected. Were you correct or incorrect in your prediction? Explain.
	\answerspace{1.5in}
	
	\item In this experiment, what is the independent variable and what is the dependent variable?
	\answerspace{1.5in}
	

	\item What would you expect to happen to the concentration of oxygen gas in the bottle? Why? How could you test this hypothesis?
	\answerspace{1.5in}
	
	\item What are the gaseous products of cellular respiration that are released from the cells of the germinating seeds?
	\answerspace{1.5in}
	
	\item Where did the seed get the fuel (glucose) for cellular respiration?
	\answerspace{1.5in}
\end{enumerate}

\textbf{Synthesis Questions}

Use available resources to help you answer the following questions.

\begin{enumerate}
	\item Consider the following overall summary equation of the hundreds of biochemical reactions involved in aerobic cellular respiration:
\begin{equation*}
	\text{C}_6\text{H}_{12}\text{O}_6 + 6\text{O}_2 + \text{ADP} + \text{P}_\text{inorganic} \rightarrow 6\text{CO}_2 + 6\text{H}_2\text{O} + \text{ATP}
\end{equation*}
Write this equation using words instead of chemical symbols. Write an equation for this process using chemical notation.
\answerspace{3.0in}
	
	\item In this activity, a germinating seed is used as a model to represent cellular respiration in all living things. Why is it reasonable to create a model for cellular respiration?
	\answerspace{2.5in}
	
	\item What is the effect of cellular respiration on the atmosphere? How is the natural greenhouse effect influenced?
	\answerspace{2.0in}
	
	\item	What role does cellular respiration in plants and other living organisms play in the carbon cycle?
	\answerspace{2.0in}
\end{enumerate}

\newpage	
\textbf{Multiple Choice Questions}

Select the best answer or completion to each of the questions or incomplete statements below.

\begin{enumerate}
	\item Which of the following is an example of a chemical element being recombined as it passes through a link in the food web?
	\begin{enumerate}[A.]
		\item Carbon from glucose in the bean seed is released as CO2 gas into the atmosphere.
		\item Oxygen from O$_2$ in the atmosphere is incorporated into molecules such as CO$_2$ and H$_2$O in the seed.
		\item Nitrogen from the soil is stored in seed proteins.
		\item Both A and B.
	\end{enumerate}
	\item How did the energy-storage molecules in the bean seed get there?
	\begin{enumerate}[A.]
		\item The seed gathered these molecules from the environment.
		\item The plant that made the seed captured energy from sunlight through photosynthesis and stored it in the seed.
		\item The seed created these molecules on its own using sunlight.
		\item Both A and B.
	\end{enumerate}
	\item The food energy that the plant uses for cellular respiration is stored in the seed's
	\begin{enumerate}[A.] 
		\item cell wall
		\item embryo
		\item cotyledons
		\item DNA
	\end{enumerate}
	\item Beans and coal both have stored energy. Where did the energy originally come from that is stored in beans and coal?
	\begin{enumerate}[A.]
		\item From the earth's gravity
		\item From the sun's light
		\item From the heat in the earth's core
		\item From the carbon dioxide in the air
	\end{enumerate}
	\item What natural ``greenhouse gases'' were produced by the bean seeds during cellular respiration?
	\begin{enumerate}[A.]
		\item Carbon dioxide gas
		\item Oxygen gas
		\item Water vapor
		\item Both A and C are correct
	\end{enumerate}
\end{enumerate}

\end{document}
