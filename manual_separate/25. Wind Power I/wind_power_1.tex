\documentclass[english,twoside]{article}

\input{../labmanual_formatting_commands} %all general latex packages, commands, and definitions now here.

\begin{document}

\Lab{25}{Wind Power 1: Distance and Speed}

\renewcommand{\arraystretch}{1.5}
\setlength{\arrayrulewidth}{0.4mm}
\setlength{\tabcolsep}{10pt}

\newlength{\myMheight}
% Create the reference text for measures
\settoheight{\myMheight}{M}



\newcolumntype{M}[1]{>{\centering\arraybackslash}m{#1}}

\makelabheader %(Space for student name, etc., defined in master.tex or labmanual_formatting_commands.tex)

\textbf{Driving Question}

How does wind speed impact a turbine's voltage output?

The amount of power generated by a wind turbine depends strongly on the speed and consistency of the wind. Faster wind speeds cause the blades to spin more quickly, increasing the generator's rotational speed and, in turn, the electrical voltage produced. However, the relationship is not perfectly linear---beyond a certain point, turbines are designed to limit rotation to prevent mechanical damage.

When selecting locations for wind energy sites, engineers must consider several key factors. The area should experience steady winds throughout the year, have stable ground for construction, and allow easy access for maintenance. Turbines must also be close enough to existing transmission lines to connect efficiently to the power grid. These factors, along with local climate, topography, and environmental impact, determine whether a site is suitable for wind energy generation.

\textbf{Materials}
\begin{itemize}
	\item Wind turbine kit	
	\item Box fan, 3 or more speeds
	\item Voltage sensor with red and black banana plug leads
	\item Alligator clip leads (2), green and black	
	\item Masking tape
	\item Meter stick
	\item Textbooks for weight (2)
\end{itemize}

\textbf{Safety}

\begin{itemize}
	\item Wear safety goggles throughout the experiment.
	\item Tie back long hair, remove dangling jewelry, secure loose clothing, and roll up long sleeves.
	\item Make sure blades are properly inserted in the turbine and screws are secure before turning on the fan.
\end{itemize}

\textbf{Consider}

You will use a fan as a wind source. Predict the fan distance and speed that will produce the highest voltage:

\begin{enumerate}
	\item The best fan distance will be:
	\begin{enumerate}[A.]
		\item Far from the turbine (1 m or more)
		\item Close to the turbine (0.5 m or less)
		\item Medium distance from the turbine (at or near 0.5 m)
	\end{enumerate}

	\item The best fan speed will be:
	\begin{enumerate}[A.]	
		\item Low
		\item Medium
		\item High
	\end{enumerate}
\end{enumerate}
\newpage
\textbf{Investigate}

\begin{enumerate}
	\item Open \textit{SPARKvue} and build a page with one graph.
	\item Connect the voltage sensor.
	\item Display Voltage on the $y$-axis and Time on the $x$-axis. 
	\item Assemble a 6-blade turbine with long blades and long tower according to the instruction manual included with the turbine kit.
 
\begin{center}
    \includegraphics[width=0.90\linewidth]{setup_1.jpg}
  \end{center}
	\item The leaf logo on the blades must face the fan. Blades can be adjusted to different angles, but for now, adjust each blade to the $20^\circ$ mark.
	\item Place the turbine close to the fan without touching it. Make sure turbine blades are exactly parallel with the fan blades as shown.
	\item Set textbooks on either side of the turbine base for stability.
	\item Insert banana plug leads into the voltage sensor if necessary. Use red for (+) and black for (-).
	\item Attach alligator clip leads to the motor terminals on the turbine. Clip the leads to the banana plugs as shown. Match black colors.
	\item Turn the fan on to its highest setting.
	\item Start collecting data. If the voltage reads negative, switch the alligator clips on the motor terminals. When voltage reads positive, label the motor terminal with the black alligator clip ``Black/-''.
	\item Observe the voltage readings for at least one minute to identify the highest voltage possible.
	\item Move the fan slowly away from the turbine to determine the distance that achieves the highest possible voltage.
	\item Use a 15-cm piece of masking tape to mark the turbine distance that achieves the highest voltage. Align the tape with the post as shown.
	\item Measure the distance from the fan to the masking tape. Record the distance in Table 1.
	\item Write your name and the distance on the tape. Leave space on the tape for future notes.
	\item Find the fan speed that produces the highest voltage at the optimal distance. Record the speed on the tape. Leave space on the tape for future notes.

\begin{center}
\begin{tabular}{ | M{2.5cm} | M{2.5cm} |}
 \hline
	\multicolumn{2}{|c|}{Table 1: Best Fan Distance and Speed} \\ \hline
 Criteria & Best Value \\ \hline
 Distance & \\[15pt] \hline
 Speed & \\[15pt] \hline
\end{tabular}
\end{center}
	\item Place the tape on the fan for reference in future activities.
	\item Stop collecting data and turn the fan off.
\end{enumerate}

\textbf{Analyze}

How do your fan distance and speed predictions compare to your results? Use data to support your answer.
\end{document}
