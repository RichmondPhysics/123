\documentclass[english,twoside]{article}

\input{../labmanual_formatting_commands} %all general latex packages, commands, and definitions now here.

\begin{document}

\Lab{11}{Radiation Energy Transfer}

\renewcommand{\arraystretch}{1.5}
\setlength{\arrayrulewidth}{0.4mm}
\setlength{\tabcolsep}{10pt}


\newcolumntype{M}[1]{>{\centering\arraybackslash}m{#1}}

\makelabheader %(Space for student name, etc., defined in master.tex or labmanual_formatting_commands.tex)


\textbf{Driving Question}

Determine the effect the color of a container has on the temperature of water in the container as it is heated using radiant energy. 
\begin{itemize}
	\item What is radiant energy? 
	\item What is the relationship between an object's color and its ability to absorb energy?
	\item Does radiant energy heat all of Earth's surfaces equally?
\end{itemize}

\textbf{Background}

The Earth receives an enormous amount of radiant energy from the sun. Solar radiation is made up of the entire spectrum of electromagnetic waves. Visible light, the light that we can see, is only a tiny part of this spectrum. Other types of electromagnetic radiation produced by the sun include infrared radiation (thermal energy), microwaves, radio waves, ultraviolet light, X-rays, and gamma rays.

Incoming radiation is scattered, reflected or absorbed by the atmosphere or the Earth's surface.  The atmosphere protects us from most X-rays, gamma rays and ultraviolet radiation by reflecting these wavelengths of light back into space. The light that travels through our atmosphere is either reflected or absorbed by Earth's surface. Different surfaces absorb and reflect differing amounts of solar radiation.  The term albedo is used to compare the degree to which different surfaces reflect incoming solar radiation.  Surfaces with high albedo reflect more radiation than surfaces with low albedo. Surfaces with low albedo absorb more radiant energy than they reflect. 
\begin{center}
\includegraphics[width=0.9\linewidth]{radiation_transfer.jpg}
\end{center}
 
When surfaces absorb radiant energy they become warmer. This in turn increases their thermal energy, or total internal energy. Likewise, cooling decreases thermal energy. The total amount of energy the Earth receives is in equilibrium with the total amount of energy the Earth loses and is called Earth's energy budget.

\textbf{Materials and Equipment}
\begin{itemize}
	\item Temperature probe (2)
	\item Radiation can (2), 1 black, 1 silver
	\item Graduated cylinder, 100-mL
	\item Insulated pad (2)
	\item Heat lamp (or 150-W lamp)
	\item Ring stand 
	\item Water, room temperature, 0.5 L
\end{itemize}

\textbf{Procedure}

\begin{enumerate}
	\item Open \textit{SPARKvue} and build a page with a graph display.
	\item Connect the two temperature probes to your device. 
	\item Set the data collection system so that both temperature probes are collecting data once every five seconds.
	\item Display a graph with both Temperature readings on the $y$-axis and Time on the $x$-axis. 
	\item Place each radiation can on an insulated pad. Keep the cans away from drafts.
	\item Why are you asked to place each radiation can on an insulated pad and to keep the cans away from drafts?
	\answerspace{2.0in}
	\item Fill each can with 200 mL of room-temperature water (the cans should be the same size so that the water level in both is the same). 
	\item Put one temperature probe into the water in the black can and the other temperature probe into the water in the silver can.
	\item Place the heat lamp so it is about 20 cm in front of the two cans. Make sure the lamp is the same distance from each radiation can to ensure even heating.
	\item How do you think the change in water temperature in the black can will compare to that of the silver can? Explain your reasoning.
	\answerspace{2.0in}
	\item Turn on the lamp and start data recording.
	\item Continue recording data for 20 minutes. Note: If necessary, adjust the scale of the graphs to show all data.
	\item How will you know which can absorbed the most radiation? 
	\answerspace{1.5in}
	\item What surfaces on Earth could the black can represent?
	\answerspace{1.5in}
	\item What surfaces on Earth could the silver can represent?
	\answerspace{1.5in}
	\item Stop data recording.
	\item Save your experiment and clean up according to your teacher's instructions.
\end{enumerate}

\textbf{Data Analysis}
\begin{enumerate}
	\item Use the graph of Temperature versus Time to determine the initial temperature, final temperature, and change in temperature for each radiation can and record the answers in Table 1.

\begin{center}
\begin{tabular}{| M{2.0cm} | M{3.5cm} | M{3.5cm} | M{4.0cm} | }
 \hline
 \multicolumn{4}{|c|}{Table 1: Recorded and calculated temperatures} \\ \hline
 & Initial Temperature \newline ($^\circ$C) & Final Temperature \newline ($^\circ$C) & Change in Temperature \newline ($^\circ$C) \\ \hline
 Silver can & & & \\[15pt] \hline
 Black can & & & \\[15pt] \hline
\end{tabular}
\end{center}	
\newpage		
	\item Sketch or print a copy of the graph of Temperature versus Time. Include the data for both radiations on the same set of axes. Label each trial as well as the overall graph, the $x$-axis, and the $y$-axis, and include numbers on the axes.
\begin{center}
\includegraphics[width=0.7\linewidth]{blank_graph.png}
\end{center}
\end{enumerate}
\textbf{Analysis Questions}
\begin{enumerate}
	\item Examine your Temperature versus Time graph and Table 1. Which can absorbed more radiant energy? Use your data to support your answer.
	\answerspace{2.5in}
	
	\item Compare the slope of data collected for the black can to the slope of the data collected for the silver can. What does this tell you about the efficiency of the black can's ability to absorb radiant energy? 
	\answerspace{2.5in}
	
	\item What is the relationship between the color of an object and the object's ability to absorb heat?
	\answerspace{1.5in}
	
	\item Does radiant energy affect all Earth's surfaces equally? Use your data to support your answer. 
	\answerspace{2.5in}
\end{enumerate}
\newpage	
\textbf{Synthesis Questions}

Use available resources to help you answer the following questions.

\begin{enumerate}
	\item Suppose you had to choose a roof color for a new house and were given two choices: dark grey or light grey. Which would you choose to keep the house cooler in the summer? Why?
	\answerspace{2.5in}
	
	\item On a sunny summer day would you expect an asphalt street or a cement driveway to feel hotter? Explain.
	\answerspace{2.5in}
	
	\item Would you expect the albedo of a mountain range to change after the first snowfall? Explain.  
	\answerspace{2.2in}
\end{enumerate}
	
\textbf{Multiple Choice Questions}

Select the best answer or completion to each of the questions or incomplete statements below.

\begin{enumerate}
	\item What is the primary source of radiant energy for the Earth?
	\begin{enumerate}[A.]
		\item Earth's moon
		\item The oceans
		\item The sun
		\item Electricity
		\item None of the above
	\end{enumerate}
	\item Which of the following most accurately describes what happens to incoming solar radiation when it reaches Earth?
	\begin{enumerate}[A.]
		\item It is reflected
		\item It is absorbed
		\item It is scattered
		\item It is reflected, absorbed, and scattered
		\item None of the above
	\end{enumerate} 
	\item Which of the following surfaces has the highest albedo? 
	\begin{enumerate}[A.]
		\item Dark colored rocks
		\item Grass
		\item Soil
		\item Snow
		\item Pavement
	\end{enumerate}
	\item Which can do you expect will absorb more radiant energy under sunlight?
	\begin{enumerate}[A.]
		\item A black can 
		\item A white can
		\item A shiny metallic can
		\item A yellow can
		\item A green can
	\end{enumerate} 
	\item What process causes an object's temperature to increase the most?  
	\begin{enumerate}[A.]
		\item Scattering radiant energy 
		\item Reflecting radiant energy 
		\item Absorbing radiant energy
		\item Emitting radiant energy
		\item Transferring radiant energy
	\end{enumerate}
\end{enumerate}


\end{document}

