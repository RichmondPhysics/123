\documentclass[english,twoside]{article}

\input{../labmanual_formatting_commands} %all general latex packages, commands, and definitions now here.

\begin{document}

\Lab{12}{Greenhouse Gases}

\renewcommand{\arraystretch}{1.5}
\setlength{\arrayrulewidth}{0.4mm}
\setlength{\tabcolsep}{10pt}


\newcolumntype{M}[1]{>{\centering\arraybackslash}m{#1}}

\makelabheader %(Space for student name, etc., defined in master.tex or labmanual_formatting_commands.tex)


\textbf{Driving Question}

How do greenhouse gases interact with the atmosphere and cause a temperature increase?
\begin{itemize}
	\item Can we reproduce a greenhouse environment and study the effect of an introduced harmful gas?
	\item What are some possible solutions to global warming and ozone-depletion?
\end{itemize}

\textbf{Background}

Carbon dioxide (CO$_2$) and methane (CH$_4$) are classified as greenhouse gases---atmospheric gases that absorb infrared radiation emitted by the Earth's surface, trapping heat within the atmosphere. Incoming solar radiation passes through the atmosphere and is absorbed by the Earth's surface. This absorbed energy is then reradiated as longwave infrared (IR) radiation. While some of this IR escapes into space, greenhouse gases absorb a significant portion, thereby warming the atmosphere in a process known as the \textit{greenhouse effect}. Without these gases, Earth's surface would be significantly colder and largely uninhabitable, making life as we know it impossible.

In addition to CO$_2$ and CH$_4$, several other greenhouse gases exhibit even greater heat-trapping capacities. Chlorofluorocarbons (CFCs), for instance, were widely used in the 20th century as refrigerants and propellants in aerosol products. Although effective in these roles, CFCs were found to contribute substantially to stratospheric ozone (O$_3$) depletion by reacting with and binding to oxygen atoms. As a result, many CFCs were phased out and replaced with alternative compounds such as difluoroethane (C$_2$H$_4$F$_2$) and tetrafluoroethane (C$_2$H$_2$F$_4$). While these substitutes do not deplete the ozone layer to the same extent as CFCs, they are potent greenhouse gases with high global warming potentials.

It is important to distinguish between ozone depletion and global warming, as they are separate environmental concerns. Ozone depletion primarily poses biological risks, particularly increased exposure to harmful ultraviolet (UV) radiation, which can lead to higher rates of skin cancer and negatively affect polar ecosystems. In contrast, global warming is concerned with the rise in Earth's average temperature due to the accumulation of greenhouse gases in the atmosphere.

A compound can contribute to ozone depletion, global warming, both, or neither. For example, difluoroethane and tetrafluoroethane do not significantly deplete ozone but have a high capacity to trap heat in the atmosphere. According to the Intergovernmental Panel on Climate Change (IPCC, 1995), difluoroethane has a global warming potential (GWP) of approximately 1800. The GWP is a relative measure that compares the warming effect of a compound to that of CO$_2$, which is assigned a baseline value of 1.

Data from the National Oceanic and Atmospheric Administration (NOAA) indicate that global surface temperatures increased by approximately 0.5$^\circ$C to 0.9$^\circ$C over the 20th century. Moreover, over the past 50 years, the rate of warming has accelerated to about 0.13$^\circ$C per decade. Although these changes may seem modest, they are both statistically significant and environmentally impactful.

\textbf{Materials and Equipment}
\begin{itemize}
	\item Data collection system	
	\item Heating lamp 
	\item Fast-response temperature probe 	
	\item Ring stand
	\item EcoChamber with stoppers	
	\item Balance (1 per class)
	\item Size 5 or 5 1/2 solid stoppers (2)	
	\item Canned keyboard duster (fresh)
	\item Dark aquarium rocks or dark sand (approximately 200 g)
	\item CO$_2$ gas sensor		
	\item Heavy-duty tape
	\item Dry ice
\end{itemize}

\textbf{Safety}
\begin{itemize}
	\item Pressurized cans of difluoroethane can become very cold, especially when inverted, and may burn your hands or even cause frostbite if propellant is continually released for long periods of time. Activate the can in short bursts only. Hold the can upright, and do not shake the can.
	\item Inhaling excessive concentrations of difluoroethane causes dizziness and can be fatal. Avoid direct inhalation. In addition, it is flammable. Use in a well-ventilated area and follow all safety precautions printed on the can.
\end{itemize}

\textbf{Procedure}
\begin{enumerate}
	\item Open \textit{SPARKvue} and build a page with a graph display.
	\item Connect the fast-responses temperature sensor to the data collection system. 
	 \item Display Temperature on the $y$-axis of a graph with Time on the $x$-axis. 
	\item Set up the EcoChamber as indicated in the following steps:
	\begin{enumerate}[(a)]
		\item Place 3 flat stoppers into the holes on the top of each EcoChamber. These stoppers have small holes to accommodate the temperature sensor. Plug two of these holes with the small rubber dowels. 
		\item Thread the temperature sensor through the hole in the last stopper on the lid of the chamber. Pull the temperature sensor through the stopper until the sensor hangs down approximately halfway in the chamber.
		\item Once the temperature sensor is in place, cover the hole with heavy-duty tape. 
		\item Measure out approximately 200 grams of aquarium rocks or enough to cover the bottom of each chamber. Place these in the chamber.
		\item Place the lid on the chamber, and stopper the holes on the sides of the chamber with solid stoppers (size 5 or 5 1/2 will work). If the stoppers are not solid, cover them with heavy-duty tape to ensure a good seal. 
	\end{enumerate}
	\item Why do you think the holes in the sides need to be sealed, as well as the extra hole in the top? 
	\answerspace{2.0in}
	\item Position the heating lamp so that it will shine on the chamber, angled slightly downward to increase the amount of solar radiation hitting the rocks. Do not turn the light on yet. Your setup should look like the figure below.
	\begin{center}
\includegraphics[width=0.45\linewidth]{setup_1.png}
\end{center}

	\item Turn on the lamp and begin recording data. 
	\item After 5 minutes, turn the lamp off and continue to record data for 5 minutes more.
	\item Stop recording data. 
	\item Adjust the scale of the graph. 
	\item Open the EcoChamber and allow it to cool completely. You may want to replace the rocks with room temperature rocks, but use the same mass of rocks as you did before.
	\item If you decide to replace the rocks, why is it necessary to use the same amount of rocks as you used the first time?
	\answerspace{2.0in}
	\item Replace the lid on the EcoChamber. Ensure that the temperature sensor is hanging as it was in the first trial, and that the lamp and the chamber are positioned exactly as they were in the first trial.
	\item Place the plastic straw that accompanies the keyboard duster into the nozzle of the can. Do not shake the can. When the trigger is pulled, the propellant should leave the can in a steady, concentrated stream.
	\item Peel back the tape on the rubber stopper on the side of the chamber and place the straw of the keyboard duster into the hole. Fill the chamber with difluoroethane by pulling the trigger on the can in a series of short bursts. Keep the can upright while dispensing.
	\begin{center}
\includegraphics[width=0.45\linewidth]{setup_2.png}
\end{center}
 	\item Begin recording data without turning on the lamp and continue to dispense the difluoroethane in short bursts.
	\item Watch the data carefully. Once the temperature inside the chamber is below the starting temperature of the control run, stop dispensing difluoroethane.
	\item Remove the straw and immediately plug the hole. 
	\item Watch the temperature on the graph. When the temperature is just below the starting point of the first run, stop recording data. 
	\item Why is it necessary to wait for the experimental chamber to reach room temperature, or at least the same temperature as the control chamber?
	\answerspace{2.0in}
	\item Hide this last run of data. You will not need it.
	\item Turn on the light and begin data recording. 
	\item Collect data for 5 minutes under the lamp. Then, turn off the light and continue to collect data for an additional 5 minutes while the chamber cools.
	\item Stop data recording. 
	\item Adjust the scale of the graphs if necessary. 
	\item Save your experiment. 
\end{enumerate}

\textbf{Data Analysis}
	\begin{enumerate}
		\item Find the initial, final, maximum, and change in temperature for both chambers. Record this data in Table 1.

\begin{center}
\begin{tabular}{ | M{3.0cm} | M{2.5cm} | M{2.5cm} | M{2.5cm} | M{2.5cm} | }
 \hline
 \multicolumn{5}{|c|}{Table 1: Temperature data} \\ \hline
 Chamber & Initial Temp. \newline ($^\circ$C) & Maximum Temp. ($^\circ$C) & Increase in Temp. ($^\circ$C) & Change in Temp. ($^\circ$C) \\ \hline
 Control (air) & & & & \\[15pt] \hline
 Experimental (difluoreothane) & & & & \\[15pt] \hline
\end{tabular}
\end{center}
	\item Which system retained heat longer? How do you know?
	\answerspace{2.0in}
	\item How did the change in temperature from initial to final temperature for the experimental run compare to the change in temperature for the control run?
	\answerspace{1.5in}
\end{enumerate}
\newpage
\textbf{Analysis Questions}
\begin{enumerate}
	\item How significant are the differences that you observed in heat retention and maximum temperature?
	\answerspace{2.0in}
	\item In analyzing this data, which of the following is more valuable to compare: the overall change in temperature, the heating change in temperature, the cooling change in temperature, or the difference in maximum temperatures? Explain your reasoning.
	\answerspace{3.0in}
	\item In what ways can you use the results from this demonstration to predict the effects of this gas on the atmosphere?
	\answerspace{2.0in}
	\item In what ways does this demonstration fail to predict what effect this gas would have on the atmosphere?
	\answerspace{2.0in}
\end{enumerate}

\textbf{Synthesis Questions}

Use available resources to help you answer the following questions.

\begin{enumerate}
	\item Considering the severity of the IR absorption of the difluoroethane and its increased ability to trap heat, why are scientists so concerned about carbon dioxide and not gases like difluoroethane and other man-made gases?
	\answerspace{3.0in}
	\item In some cases, one solution to an environmental problem can result in another environmental problem. In this case, ozone-depleting chlorofluorocarbons (CFCs) were replaced by tetrafluoroethane, which contributes to global warming. What are some ways to avoid this situation?
	\answerspace{3.0in}
	\item At your home, examine canisters and other sources of propellants to see what, if any, greenhouse gases discussed in this lab may be contained within those canisters. Make a list of the products here, and the propellants they contain.
	\answerspace{3.5in}
\end{enumerate}
\newpage
\textbf{Multiple Choice Questions}

Select the best answer or completion to each of the questions or incomplete statements below.
\begin{enumerate}
	\item Which of the following is NOT true of chlorofluorocarbons (CFCs)? 
	\begin{enumerate}[A.]
		\item CFCs were commonly used as propellants and refrigerants 
		\item CFCs are extremely toxic to humans under nomral exposure conditions
		\item CFCs cause ozone-depletion and contribute to the growing of the hole in the ozone layer	
		\item CFCs are greenhouse gases associated with increased global warming trends
		\item All of the above are true
	\end{enumerate}
	\item Which atmospheric layer(s) are primarily affected by atmospheric pollution? 
	\begin{enumerate}[A.]
		\item Mesosphere
		\item Stratosphere
		\item Troposphere
		\item Ionosphere
		\item B and C
	\end{enumerate}
	\item What makes a gas an ozone-depleting gas?
	\begin{enumerate}[A.]
		\item The gas will re-radiate longwave infrared energy into the ozone
		\item The gas will absorb ozone molecules
		\item The gas will bind with one of the oxygen atoms in the ozone molecule, reducing it to O$_2$
		\item Ozone-depletion only occurs when temperatures rise and the atmosphere warms up
		\item Any gas can become an ozone-depleting gas if there is enough of it
	\end{enumerate}
	\item What is the main cause of atmospheric warming from greenhouse gases?
	\begin{enumerate}[A.]
		\item Greenhouse gases usually have a high specific heat and get very hot
		\item Greenhouse gases make the atmosphere much thicker and more polluted, so they trap the sun's direct rays and warm the air
		\item Greenhouse gases absorb and re-emit IR waves that enter the atmosphere from the sun
		\item Greenhouse gases absorb and re-emit IR waves that are radiated from Earth's surfaces
		\item None of the above are true
	\end{enumerate}
	\item Why is difluoroethane used in place of chlorofluorocarbons today as a propellant?
	\begin{enumerate}[A.]
		\item Difluoroethane breaks down harmful ozone molecules more rapidly than CFCs
		\item Difluoroethane has a lower global warming potential than most CFCs
		\item Although difluoroethane has a higher global warming potential than many CFCs, it was adopted primarily to protect the ozone layer
		\item Difluoroethane is safer for human health and the environment compared to CFCs
		\item None of the above are true
	\end{enumerate}
\end{enumerate}


\end{document}
