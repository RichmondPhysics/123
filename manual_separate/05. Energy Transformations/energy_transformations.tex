\documentclass[english,twoside]{article}

\input{../labmanual_formatting_commands} %all general latex packages, commands, and definitions now here.

\begin{document}

\Lab{5}{Energy Transformations}

\renewcommand{\arraystretch}{1.5}
\setlength{\arrayrulewidth}{0.4mm}
\setlength{\tabcolsep}{10pt}


\newcolumntype{M}[1]{>{\centering\arraybackslash}m{#1}}

\makelabheader %(Space for student name, etc., defined in master.tex or labmanual_formatting_commands.tex)

\textbf{Driving Question | Objective}

How are energy transformations observed?

Energy is constantly moving and changing all around you. Consider a few of the energy changes that happen when you ride a bicycle.

Your legs would not be able to push the bicycle pedals without the energy from food. Plants use photosynthesis to convert electromagnetic (light) energy into chemical energy. Plants store this chemical energy in molecules like carbohydrates, fats, and proteins found in foods you eat such as fruits, vegetables, grains, and nuts.

Muscles in your body convert chemical energy from food molecules into mechanical energy needed to push the bicycle pedals. Your body becomes warmer while pedaling because thermal (heat) energy is released during energy conversions. Chemical energy also helps your body produce electricity. Your nervous system uses electrical energy to communicate with your entire body and remind your muscles how to ride a bicycle.

\textbf{Materials}
\begin{itemize}
	\item Data collection system
	\item Temperature sensor
	\item Voltage sensor with red and black banana plug leads
	\item 250-mL Erlenmeyer flask
	\item One-hole rubber stopper
	\item Cloth towel or potholder glove
	\item 175 mL sand
	\item Alligator clip adapters (2), red and black
	\item Light sensor
	\item Solar panel	
	\item Two or more kinds of fruits or vegetables
	\item Three or more kinds of metal pieces (coins, nails, screws, paper clips, wires or strips)
	\item Sheet of white paper
\end{itemize}

\textbf{Consider}
\begin{enumerate}
	\item What type of energy is required to turn on a light bulb?
		\begin{enumerate}[A.]	
			\item Mechanical
			\item Thermal
			\item Chemical
			\item Electrical
			\item Electromagnetic
 		\end{enumerate}
	\item Provide your own example where energy is converted from one form to another.
	\answerspace{1.6in}
	\item Describe at least two types of energy conversions that occurred in your example.
	\answerspace{1.6in}
\end{enumerate}

\textbf{Investigate: Sand Energy}
\begin{enumerate}
	\item Connect the temperature sensor.
	\item Fill the flask with 175 mL of sand.
	\item Seal the flask with the stopper. Insert the temperature sensor through the stopper hole so the bottom of the probe is below sand level.
	\item Will temperature change if you shake the flask vigorously for one minute? How much? Draw your prediction on the graph. Add Time to the $x$-axis and Temperature to the $y$-axis.

\begin{center}
\begin{minipage}[h]{1.0\linewidth}
{\par\centering \includegraphics[width=0.55\linewidth]{blank_graph.png} \par}
\end{minipage}
\end{center}
	\vspace*{0.5in}

	\item Build a page with a graph display. Place Temperature on the $y$-axis.
	\item Grasp the flask with the glove or cloth. Start collecting data. Shake the flask vigorously for one minute.
	\item Stop collecting data after one minute. Scale the graph.
\end{enumerate}
\newpage
\textbf{Analyze: Sand Energy}
\begin{enumerate}
	\item Was your prediction supported by the results? Why or why not?
	\answerspace{2.5in}

	\item Did the sand gain energy, or did it lose energy? Use data to support your answer.
	\answerspace{2.5in}

	\item What kind of energy did you provide to the flask?
	\begin{enumerate}[A.]
		\item Mechanical
		\item Thermal
		\item Chemical
		\item Electrical
		\item Electromagnetic
	\end{enumerate}

     \item The energy you originally provided was converted into which new energy type?
	\begin{enumerate}[A.]
		\item Mechanical
		\item Thermal
		\item Chemical
		\item Electrical
		\item Electromagnetic
	\end{enumerate}
\end{enumerate}
\newpage
\textbf{Investigate: Food Energy}
\begin{enumerate}
	\item Connect the voltage sensor. Insert banana plug leads if necessary. Use red for (+) and black for (-).
	\item Attach alligator clip adapters to the voltage sensor leads. Match colors.
	\item Select any two metals and a fruit or vegetable.
	\item Insert the metals halfway into the fruit at some distance from each other. Do not allow metals to touch.
\vspace*{5mm}
{\par\centering \includegraphics{food_energy_setup.png} \par}
\vspace*{5mm}
	\item Build a page with a digits display for Voltage. Start collecting data.
	\item Attach an alligator clip to each metal and observe the voltage (electrical energy) produced.
	\item Reverse the alligator clips and observe voltage produced. Notice the voltage is the same except one arrangement produces a negative result and the opposite arrangement produces a positive result. You can ignore negative signs in this activity.
 	\item Build a page with a table display. Set up the table like the one below. Create at least four different food-metal combinations. Write your choices in the appropriate cells in the table. Leave the Voltage column blank for now.

\begin{center}
\begin{tabular}{ | M{3.0cm} | M{3.0cm} | M{3.0cm} | M{3.0cm} |}
 \hline
 \multicolumn{4}{|c|}{Table 1: Voltage produced from food} \\ \hline
 Fruit/Vegetable & Metal 1 & Metal 2 & Voltage (V) \\ \hline
 & & & \\[15pt] \hline
 & & & \\[15pt] \hline
 & & & \\[15pt] \hline
 & & & \\[15pt] \hline
\end{tabular}
\end{center}

	\item Start collecting data. Move the alligator clips to observe voltages for each combination.
	\item Record voltages in the table.
	\item Stop collecting data.
\end{enumerate}
\newpage
\textbf{Analyze: Food Energy}
\begin{enumerate}
	\item What kind of energy was converted in order to produce electricity?
	\begin{enumerate}[A.]
		\item Mechanical
		\item Thermal
		\item Chemical
		\item Electrical
		\item Electromagnetic
	\end{enumerate}
	\item Did energy flow in a specific direction? Why or why not? Support your answer with data.
	\answerspace{1.3in}

	\item Which fruit-metal combination produced the greatest voltage? Why might this combination work better than others?
	\answerspace{1.6in}
\end{enumerate}

\textbf{Investigate: Light Energy}
\begin{enumerate}
	\item Connect the light sensor. Make sure the voltage sensor is still connected.
	\item Predict which source will produce greater light intensity (W/m$^2$) and voltage: indoor light or outdoor light. Note factors that influenced your predictions.
	\answerspace{1.3in}

	\item Attach alligator clips from the solar panel to the voltage sensor. Match black wires.
	\item Move the solar panel, sensors, and piece of paper to a location that best represents average light level inside the classroom. Avoid shadows on the paper.
	\item Set the piece of paper on the floor. Position the solar panel next to the paper. Hold the light sensor 10 cm above the center of the paper.

{\par\centering \includegraphics{light_energy_setup.png} \par}
	\item Start collecting data. Record the light intensity reading and voltage reading when each becomes stable. If the light sensor indicator is flashing green, select a higher setting. 

\textbf{Light Intensity and Voltage for Indoor Light:} \rule{4cm}{0.15mm} 	

	\item Take the paper, solar panel, and sensors outdoors. Arrange the paper, panel, and sensors on a part of the ground that represents average light level. Avoid shadows. If you must stay indoors, find a location where sunlight is available.
	\item Change the light sensor setting if the green light is flashing. Record the light intensity and voltage levels for outdoor light.

\textbf{Light Intensity and Voltage for Outdoor Light:} \rule{4cm}{0.15mm} 	

	\item Stop collecting data.
\end{enumerate}

\textbf{Analyze: Light Energy}
\begin{enumerate}
	\item Were your predictions correct? Why or why not?
	\answerspace{1.6in}
	\item What kind of energy was converted to produce electricity in the light energy activity?
	\begin{enumerate}[A.]
		\item Mechanical
		\item Thermal
		\item Chemical
		\item Electrical
		\item Electromagnetic
	\end{enumerate}
	\item Out of the three activities you performed today, which appears to have the greatest ability to produce voltage? Use data to support your answer.
	\answerspace{2.5in}
\end{enumerate}

\textbf{Extend}
\begin{enumerate}
	\item Choose any one of the three activities that most interests you. Write a testable question to investigate factors that affect the voltage produced in that system. Design an experiment to answer your testable question.
\end{enumerate}

\end{document}
