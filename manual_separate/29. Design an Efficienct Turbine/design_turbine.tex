\documentclass[english,twoside]{article}

\input{../labmanual_formatting_commands} %all general latex packages, commands, and definitions now here.

\begin{document}

\Lab{29}{Design an Efficient Turbine}

\renewcommand{\arraystretch}{1.5}
\setlength{\arrayrulewidth}{0.4mm}
\setlength{\tabcolsep}{10pt}

\newlength{\myMheight}
% Create the reference text for measures
\settoheight{\myMheight}{M}

\newcolumntype{M}[1]{>{\centering\arraybackslash}m{#1}}

\makelabheader %(Space for student name, etc., defined in master.tex or labmanual_formatting_commands.tex)

\textbf{Driving Question}

Can a different turbine blade design perform as well as or better than the plastic blades provided? 

In this activity, your goal is to create the most efficient turbine blades possible.

The blade engineering process will require applying what you learned about effective blade design and how to create the most optimal wind testing conditions with a fan. Trial-and-error testing will help you discover ways to improve the initial design of your product. Do not be discouraged if your first design fails to meet your expectations. When creating new products engineers may complete many cycles of the design process before they achieve their goals for the product.

\textbf{Materials}
\begin{itemize}
	\item Wind turbine
	\item Voltage sensor with red and black banana plug leads
	\item Current sensor with red and black banana plug leads
	\item Alligator clip adapters (2, black)
	\item Alligator clip leads (2, black and green)
	\item Box fan, 3 or more speeds (same fan as previous activity, with tape)
 	\item 33-$\Omega$ resistor
	\item Blade adapters (6)
	\item Dowels (6)
	\item Textbooks for weight (2)
	\item Disposable foil pans (2)
	\item Scissors
	\item Duct tape
	\item Hydroelectric kit
\end{itemize} 

\textbf{Safety}

\begin{itemize}
	\item Wear safety goggles throughout the experiment.
	\item Tie back long hair, remove dangling jewelry, secure loose clothing, and roll up long sleeves.
	\item Always make sure blades are properly inserted in the turbine and screws are secure before turning on the fan.
\end{itemize}

\textbf{Consider}
\begin{enumerate}
	\item Transfer the optimal turbine blade data from your piece of tape to the data table.

	\item Brainstorm ideas to improve performance for each turbine factor. You will apply your ideas to create lightweight metal turbine blades. You may not use the plastic blades in your design, but they are available for you to inspect. Pick up and explore your materials to help you think of ways to design new blades. Use only the materials supplied and one foil pan for your first design. Use textbooks to stabilize the base.

	\item Enter your best idea for each turbine factor in Table 1.
 
\begin{center}
\begin{tabular}{ | M{2.5cm} | M{2.5cm} | M{5.0cm} |}
 \hline
	\multicolumn{3}{|c|}{Table 1: Turbine Design Plan} \\ \hline
 Turbine Factor & Optimal Setting & What new design will you try? \\ \hline
 Distance from fan & &\\[35pt] \hline
 Blade length & & \\[35pt] \hline
Number of blades & & \\[35pt] \hline
Blade pitch/angle & & \\[35pt] \hline
Blade shape & & \\[35pt] \hline
\end{tabular}
\end{center}
\end{enumerate}

\textbf{Investigate}
\begin{enumerate}
	\item Get instructor approval for your plan before moving on to Step 2.
	\item Open \textit{SPARKvue} and build a page with two graphs.
	\item Connect the wireless voltage and current sensors.
	\item Display Voltage (V) on the $y$-axis of one graph and Current (mA) on the $y$-axis of the other graph.
	\item Attach alligator clip leads to the motor terminals. Connect the resistor and voltage and current sensors to the turbine.
	\item	Remove the turbine cap and loosen the wing nut. Remove the plastic turbine blades.
	\item Use the dowels, blade adapters, scissors, one foil pan, and tape to create new turbine blades.
	\item Insert the new blades into the turbine and secure all parts. Place textbooks on the base.
	\item Get instructor approval before moving on.
	\item Turn on the fan to the setting designated by your instructor.
	\item Start collecting data. Observe the highest voltage and current reading over 1 minute. Enter the highest observed voltage and current for Trial 1 in Table 2.

\begin{center}
\begin{tabular}{ | M{2.5cm} | M{2.5cm} | M{2.5cm} | M{2.5cm}|}
 \hline
	\multicolumn{4}{|c|}{Table 2: Results} \\ \hline
 Trial \# & Voltage (V) & Current (mA)  & Power (mW) \\ \hline
	1 & & & \\[15pt] \hline
	2 & & & \\[15pt] \hline
	3 & & & \\[15pt] \hline
\end{tabular}
\end{center}

	\item Turn the fan off.
	\item Calculate the power. Enter your answer in Table 2.
	\item In Table 3, enter your observations for Trial 1 and your ideas for revisions.

\begin{center}
\begin{tabular}{ | M{2.5cm} | M{2.5cm} | M{5.0cm} |}
 \hline
	\multicolumn{3}{|c|}{Table 3: Trial 1 Results and Revisions} \\ \hline
 Turbine Factor & Results & Next Revisions \\ \hline
 Distance from fan & &\\[35pt] \hline
 Blade length & & \\[35pt] \hline
Number of blades & & \\[35pt] \hline
Blade pitch/angle & & \\[35pt] \hline
Blade shape & & \\[35pt] \hline
\end{tabular}
\end{center}

	\item Get instructor approval for the revisions, then make the revisions.
	\item Get instructor approval again after making revisions.
	\item Repeat Steps 11-14.
	\item Record observations and revision ideas for Trial 2 in Table 4.

\begin{center}
\begin{tabular}{ | M{2.5cm} | M{2.5cm} | M{5.0cm} |}
 \hline
	\multicolumn{3}{|c|}{Table 4: Trial 2 Results and Revisions} \\ \hline
 Turbine Factor & Results & Next Revisions \\ \hline
 Distance from fan & &\\[35pt] \hline
 Blade length & & \\[35pt] \hline
Number of blades & & \\[35pt] \hline
Blade pitch/angle & & \\[35pt] \hline
Blade shape & & \\[35pt] \hline
\end{tabular}
\end{center}
		
	\item Check with your instructor to see if you need to complete and test the design revisions. If you complete a third trial, get instructor approval and repeat Steps 11-14. Create a table like Tables 2 and 3 on a separate piece of paper and record results and revisions for Trial 3.
\end{enumerate}

\textbf{Analyze}
\begin{enumerate}
	\item Review your results. Does it appear the optimal distance, angle, length, and shape settings for the plastic blades also work for the lightweight metal blades you designed? Why or why not?
	\answerspace{1.5in}

	\item Did your revisions improve turbine performance? Why or why not?
	\answerspace{1.5in}

	\item Estimate the number of revisions you think it would take to produce a turbine blade you are satisfied with. What is this number based on?
	\answerspace{1.5in}

	\item Which part of turbine design did you find the most challenging? What would make it easier to meet this challenge?
	\answerspace{1.5in}
\end{enumerate}

\textbf{Extend}
\begin{enumerate}
	\item Your final goal is to explore hydroelectric power as a renewable source of energy. Design hydroelectric turbine blades to be attached to the motor where voltage output can be measured with the voltage sensor. Your water source for the water-powered turbine will be a classroom faucet. Get instructor approval before performing your experiment.

Compare results from solar, wind, and hydroelectric power. Write a recommendation to adopt one of the three sources of alternative energy that is the most reliable and environmentally-friendly power source for a small device such as an LED light. Support your recommendation with data.
\end{enumerate}

\end{document}



