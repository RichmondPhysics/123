\documentclass[english,twoside]{article}

\input{../labmanual_formatting_commands} %all general latex packages, commands, and definitions now here.

\begin{document}

\Lab{24}{Generators}

\renewcommand{\arraystretch}{1.5}
\setlength{\arrayrulewidth}{0.4mm}
\setlength{\tabcolsep}{10pt}

\newlength{\myMheight}
% Create the reference text for measures
\settoheight{\myMheight}{M}

\newcolumntype{M}[1]{>{\centering\arraybackslash}m{#1}}

\makelabheader %(Space for student name, etc., defined in master.tex or labmanual_formatting_commands.tex)


\textbf{Driving Question}

\begin{wrapfigure}{r}{0.5\textwidth}
  \begin{center}
    \includegraphics[width=0.48\textwidth]{generator.png}
  \end{center}
\end{wrapfigure}


How is electrical energy made? Electricity is seen in nature on a large scale as lightning, but lightning is not "harvested" to light buildings, run air conditioning, or power computers and mobile devices. Instead, electricity has to be generated by converting it from another form of energy.
   
A device called a generator is often used in the process of converting energy from one form to another. Generators can vary in size and shape, but all generators are composed of a few essential parts: a coil of wire, one or more magnets, and a frame to hold everything in place. To produce electricity, the generator is designed so the magnet and/or coil moves (either the coil is stationary and the magnet moves, or the magnet is stationary and the coil spins). 

In this activity, you will build a your own generator and then explore how different variables affect how much electricity you can produce.

\textbf{Objectives}
\begin{itemize}
	\item Generate electricity using magnets and coils of wire.
	\item Compare the voltages produced by spinning magnets within different numbers of turns of wire.
\end{itemize}

\textbf{Materials}
\begin{itemize}
	\item Chromebook, computer, or mobile device
	\item Graphical Analysis 4 app
	\item Go Direct Energy
	\item SimpleGEN Kit or strong magnets, magnet wire (enameled copper wire), axle rod, open box (cardboard or PVC) with holes for axle, and sandpaper
	\item 100 $\Omega$ resistor
	\item 2 wires with clips
	\item drill
	\item small incandescent light bulb (holiday-light style)
	\item red LED
	\item tape
\end{itemize}
\newpage
\textbf{Preliminary Questions}
\begin{enumerate}
	\item What are the primary sources of energy that are converted to electricity in the United States? Can you think of some sources that are not thermally driven and do not require the burning of fuels?
	\answerspace{2.5in}

	\item What components make up a typical electrical generator?
	\answerspace{2.0in}

	\item What are some variables that may affect generator performance?
	\answerspace{2.0in}

	\item What are some of the ways we can move magnets and coils relative to one another? 
	\answerspace{2.0in}
\end{enumerate}

\textbf{Procedure}

\textit{Part I: Preliminary Activity}
\begin{enumerate} 
	\item Set up the equipment.
	\begin{enumerate}[(a)]
		\item Tape one end of the wire to the housing.
		\item Wrap the wire so it creates a clean coil, winding the wire in the same direction each time (see Figure 1). Note: Your instructor will tell your group how many ``wraps'' or ``windings'' of wire to use on your generator. Different groups will use a different number of windings. 
		\item Sand the ends of the wire until they are a bright copper color.
		\item Insert the magnets into the magnet holder.
		\item Position the magnet holder inside the housing and slide the rod through the housing and magnet holder so the magnet holder can spin freely.
		\item Connect the red LED to the free ends of the coil.
	\end{enumerate}
	\item Spin the axle by hand so that the magnet assembly turns inside the coils of wire. Does the LED bulb light when you spin the magnet? Replace the LED bulb with the small incandescent holiday-light style bulb. Can you light the incandescent bulb? Record your answer in the data table. 
	\item If you have access to a drill, connect the drill chuck to the axle rod. Spin the drill, starting slowly. Can you light either bulb using the drill? Record your answer in the data table.
\end{enumerate}

\textit{Part II: Quantitative Analysis}
\begin{enumerate}
	\setcounter{enumi}{3}
	\item Open \textit{SPARKvue} and build a page with one graph.
	\item Connect the voltage sensor.
	\item Display a graph with Voltage on the $y$-axis and Time on the $x$-axis.
	\item Set the switch on the voltage sensor to External Load.
	\item Set up the data-collection mode. 
	\begin{enumerate}[(a)]
		\item Click or tap Mode to open Data Collection Settings. 
		\item Change Rate to 60 samples/s. Click or tap Done. 
	\end{enumerate}
	\item Connect the red and black voltage sensor source wires to your copper coil. Make sure the metal clips are attached to the area of the copper wire that you sanded in Step 1c. 
	\item Connect the voltage sensor external load terminals to a 100 $\Omega$ resistor. 
	\item Click or tap Collect to start data collection. 
	\item Collect data for 30 seconds. Spin your generator by hand several times over the course of data collection. 
	\item When data collection is complete, examine your graph of Voltage vs. Time. Click or tap graph Tools and choose View Statistics. Record the maximum and minimum voltage in the data table. 
	\item If you have access to a drill, repeat Steps 11-13, spinning the generator with the drill instead of by hand.
\end{enumerate}
 
\textbf{Data}
Number of coil windings: \rule{4cm}{0.15mm} 
 	
\begin{center}
\begin{tabular}{ | M{4.5cm} | M{2.5cm} | M{2.5cm} | }
 \hline
 & Hand-spin trial & Drill-spin trial \\ \hline
Light LED bulb & & \\[15pt] \hline
 Light incandescent bulb & & \\[15pt] \hline
 Minimum voltage recorded (V) & & \\[15pt] \hline
 Maximum voltage recorded (V) & & \\[15pt] \hline
\end{tabular}
\end{center}

\textbf{Processing the Data}

Share your results with the class to compare the generators used by each group. 

\textbf{Analysis Questions}
\begin{enumerate}
	\item Does the number of coil windings affect the voltage output of the generator? Support your answer with your data.
	\answerspace{1.5in}
	\item Does the speed at which the magnets rotate affect the voltage output of the generator? Support your answer with your data.
	\answerspace{1.5in}
	\item What other factors may affect the power output of an electric generator?
	\answerspace{1.5in}
	\item If you were able to light the LED bulb, why did it flicker on and off?
	\answerspace{1.5in}
\end{enumerate}

\textbf{Extend}
\begin{enumerate}
	\item Using your own materials, make a homemade generator that will light more than one bulb wired in series. Document your project and compare it to the generators made in this exploration.	
	\item Turn your generator into a motor.
	\item Turn your generator into a wind turbine. Can you get your device to spin and generate electricity when the wind blows on it?
	\item Test generators with smaller or larger gauge wire. Does it affect voltage output of the generator? 
	\item Test different magnets in the generator. How does that affect voltage output?
\end{enumerate}

\end{document}
