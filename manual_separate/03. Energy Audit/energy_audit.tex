\documentclass[english,twoside]{article}

\input{../labmanual_formatting_commands} %all general latex packages, commands, and definitions now here.

\renewcommand{\arraystretch}{1.5}
\setlength{\arrayrulewidth}{0.2mm}
\setlength{\tabcolsep}{10pt}

\begin{document}

\Lab{3}{Energy Audit}

\renewcommand{\arraystretch}{1.5}
\setlength{\arrayrulewidth}{0.4mm}
\setlength{\tabcolsep}{10pt}


\newcolumntype{M}[1]{>{\centering\arraybackslash}m{#1}}

\makelabheader %(Space for student name, etc., defined in master.tex or labmanual_formatting_commands.tex)

\textbf{Driving Question}

We all use energy to turn on lights, heat and cool our homes, get to school, and power our electronics. In the United States, energy consumption is broken down into four sectors: residential, commercial, industrial, and transportation.
\begin{itemize}
	\item Residential includes energy used in places like houses and apartment buildings.
	\item Commercial accounts for energy used in public spaces such as office buildings, schools, and hospitals.
	\item Industrial includes energy to grow and make goods such as food, cars, and buildings.
	\item Transportation includes the gasoline and fuel used to drive cars and fly planes.
\end{itemize}

\begin{center}
\begin{figure}[h]
{\par\centering \includegraphics[width=0.25\linewidth]{energy_sectors_us_2024.png} \par}
\caption{Share of total energy consumed by major sectors in the United States, 2024} \label{US_energy_sectors_2024}
\end{figure}
\end{center}

Conducting an energy audit will allow you to better understand how much energy is used by the devices in your home and classroom. Based on this information, brainstorm ways to reduce the amount of energy you use.\footnote{US Energy Information Administration, Monthly Energy Review \url{www.eia.gov/totalenergy/data/monthly/}}

\textbf{Objectives}
\begin{itemize}
	\item Measure electricity usage by several devices in your classroom and at home.
	\item Make a plan for how you will conduct a home energy audit.
	\item Calculate energy usage per person in your home.
	\item Consider ways to conserve energy at home and school.
	\item Determine ways your classroom, school, or home could become more efficient.
\end{itemize} 

\textbf{Materials}
\begin{itemize}
	\item Data collection system
	\item Temperature sensor
	\item Multiple devices that use electricity
\end{itemize}
 
\textbf{Consider}

\begin{enumerate}
	\item List as many devices in your classroom that are using electricity as you can. 
	\answerspace{2.4in}

 	\item What is ``phantom'' power? What types of phantom power use are occurring in your classroom right now? 
	\answerspace{2.4in}

	\item What is the difference between energy efficiency and energy conservation? 
	\answerspace{2.4in}
\end{enumerate}

\newpage

\textbf{Part I: Classroom Energy Audit}

\textbf{Investigate}
  \begin{enumerate}
	\item Perform a general assessment of the classroom by answering the following questions.
	\begin{itemize} 
		\item How many devices are plugged in?
		\answerspace{1.5in}

		\item How many lights are on? What types of light bulbs are in use?
		\answerspace{1.5in}

		\item How high are the ceilings?
		\answerspace{0.6in}

		\item What source(s) of energy are used by the university?
		\answerspace{1.0in}
	\end{itemize}
	\newpage
	\item Examine several devices in your classroom to find their energy information. Record the voltage and current, and the power if available. Use this information to determine how much energy is consumed by these devices in a given amount of time.
\newpage


	\item Use a temperature sensor to measure the temperature in various parts of your classroom.
	\begin{itemize}
		\item Near the doors (front, back, and storage room) 
		\answerspace{1.5in} 
		\item Near the windows
		\answerspace{0.8in}
		\item Close to the ceiling
		\answerspace{0.8in}
	\end{itemize}
	
\textbf{Processing Data}
 	
 	\item For three of the devices, calculate how much energy is consumed during the year. How could you conserve energy use for the devices?
	\answerspace{4.0in}

 	\item How does the temperature compare in different places in your classroom? What are ways to reduce energy use for heating or cooling in your classroom?
	\answerspace{1.5in}

 	\item What are ways your class could help conserve energy at the university? 
\answerspace{1.5in}
\end{enumerate}

\textbf{Part II: Home Energy Audit} 

\textbf{Investigate}
\begin{enumerate}
	\item Create a plan for conducting a home energy audit. Consider the following questions as you develop your plan.
	\begin{itemize} 
		\item How many rooms are in your home? What is the square footage? How many people live there?
		\item How many lights are in your home? What types of light bulbs are in use?
		\item What direction do the windows face? Are they single- or double-pane windows?
		\item Is temperature controlled in each room or centrally? What temperature is the thermostat set to?
		\item How high are the ceilings?
		\item What source(s) of energy are used by your home?
	\end{itemize}

	\item Determine how much energy is used by the devices in your home each month (use the table/worksheet on Blackboard). \\
\begin{comment}
Create a table modeled after the following example.

	\begin{center}
\begin{tabular}{ | M{1.6cm} | M{2.2cm} | M{4.0cm} | M{5.5cm} | }
 \hline
 Device & Energy usage (W) & Estimated usage/month (kWh) &Operating cost \newline (usage/month $\times$ cost/kWh) \newline (use $\$0.15$ if you don't know) (\$) \\ \hline
 Refrigerator & & & \\[15pt] \hline
 Computer & & & \\[15pt] \hline
 TV & & & \\[15pt] \hline
\end{tabular}
\end{center}
	\item	Use a temperature sensor to measure the temperature in various parts of your home. Record the values in a separate table.
\end{comment}

\textbf{Processing Data}
 	\item Which devices in your home use the most energy? The least? 
\answerspace{1.5in}

 	\item Calculate energy usage/month/person in your home. 
\answerspace{2.0in}

 	\item Research to find the carbon emission values for the source of energy that is used to produce the electricity that is used in your home (e.g., natural gas or coal). Calculate the carbon footprint for the devices you measured.
	\answerspace{2.0in}

 	\item What are ways can you make your home more energy efficient? 
	\answerspace{4.0in}
\end{enumerate}
\textbf{Extend}

\begin{enumerate}
%	\item Use a smart outlet to measure energy use over a 24-hour period for a single device. How does %energy use change during the day? Estimate how energy use would change over the entire year due to %varying factors such as temperature and amount of daylight.
	\item Conduct research to compare your energy usage to people in other parts of the country. What role does climate play in affecting energy use in different regions? 
	\item	Learn about ways that buildings can be designed to reduce energy consumption. Some factors to consider include: green roofs, insulation, vegetation, and paint color.
%	\item	Research to find an energy audit tool that will allow you to measure all the energy you use, %including transportation.
%	\item	Write a letter to an administrator at your school to share ideas for how your school can reduce %energy consumption and save money
	%\item	Share your results from this experiment with people living in your home. Work together to %come up with a plan for how you can reduce energy use in your home. In a month or two, repeat your %measurements and determine if there have been changes in energy consumption patterns.
	\item	Imagine you are responsible for replacing a device in your home that uses electricity, such as a TV, water heater, or light bulb. Do research to find the most energy efficient product you could buy. Is it more expensive than less efficient options? How will you balance cost with energy savings?
	\item	Learn about how energy use in your country compares to energy use in other parts of the world. 
%	\item	You have now spent a lot of time thinking about the importance of energy conservation. Imagine %that you are having a conversation with someone who does not think it is important to conserve energy. %Explain different reasons that might make them change their mind.
	\item	Does it take more energy to charge something every day (such as a laptop) or to leave it plugged in overnight? Support your answer.
\end{enumerate}

\end{document}


