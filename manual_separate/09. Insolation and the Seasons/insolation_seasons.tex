\documentclass[english,twoside]{article}

\input{../labmanual_formatting_commands} %all general latex packages, commands, and definitions now here.

\begin{document}

\Lab{9}{Insolation and the Seasons}

\renewcommand{\arraystretch}{1.5}
\setlength{\arrayrulewidth}{0.4mm}
\setlength{\tabcolsep}{10pt}


\newcolumntype{M}[1]{>{\centering\arraybackslash}m{#1}}

\makelabheader %(Space for student name, etc., defined in master.tex or labmanual_formatting_commands.tex)


\textbf{Driving Question}

The earth revolves around the sun once a year in an almost-circular (elliptical) path. The earth also rotates around an axis that is tilted about $23.5^\circ$ from a line perpendicular to the plane of its path around the sun.
\begin{itemize}
	\item How are these characteristics about the earth's movements related to the change in temperatures observed through the seasons? 
	\item What affect does the angle of solar insolation have on the amount of solar energy delivered to a given area?
	\item Why do the warmest temperatures of the year often occur a month or two after the summer solstice? 
	\item Why do the coldest temperatures of the year often occur a month or two after the winter solstice?
\end{itemize}
 
\textbf{Background}
Energy from the sun is by far the most important factor in our weather and climate. Solar radiation comprises more than $99.9\%$ of the energy that warms the earth, drives the winds, and stirs the ocean currents. The solar energy that the earth receives is called \textbf{insolation}. 

When the surface of the earth directly faces the sun at a 90-degree angle, insolation is highest. As the angle increases between the surface and the rays of sunlight, the same amount of energy is spread over a larger region and the insolation is reduced. This is known as the projection effect and is the reason why polar regions are much colder than equatorial regions. Therefore, the amount of insolation that a part of the earth receives during the day depends on the latitude at that part of the earth. 

Earth spins daily around its axis (axis of rotation), which is tilted to approximately 23.5 degrees relative to its orbit around the sun. Thus, no matter the time of year, one part of the planet is always exposed to more direct insolation than another. As the earth orbits the sun, the amount of insolation will change at any particular location, causing the seasons to change. With its elliptical orbit, the distance from the earth to the sun varies by only 5 million miles, or about $3\%$ of the average distance from the sun (the average distance from the sun is about 150 million miles) over the course of one year. The earth is closest to the sun (perihelion) around January 4th each year and furthest from the sun (aphelion) in early July.
 
At any point during the day, the amount of energy that a particular part of the earth receives changes due to the earth's rotation. One half of the earth receives sunlight, while the other half receives none. During rotation, the amount of sunlight reaching a specific location can vary due to terrain, latitude, and many other factors. 

\textbf{Materials}
\begin{itemize}
	\item Temperature sensor	
	\item Cardboard, $15 \times 15$ cm
	\item Small tripod base and rod	
	\item Black construction paper, $15 \times 15$ cm
	\item Three-fingered clamp	
	\item Drinking straw
	\item Protractor	
	\item Tape, roll
	\item Scissors	
	\item Glue, bottle
\end{itemize}

\textbf{Procedure}

\textbf{Part 1: Make the solar panel and test it at $90^\circ$ insolation}
\begin{enumerate} 
	\item Make a solar energy collection panel as follows:
	\begin{enumerate}[(a)]
		\item Glue a piece of black construction paper to the surface of the cardboard. 
		\item Tape the protractor to it so it is perpendicular to the surface of the cardboard.
	\end{enumerate}
	\begin{center}
\includegraphics[width=0.3\linewidth]{setup_1_new.jpg}
\end{center}
	\item Tape the straw to the protractor so that it is perpendicular ($90^\circ$) to the cardboard and the end of the straw is about 0.5 cm from the surface of the cardboard.
	\item Tape the temperature sensor to the cardboard with its end near the center of the cardboard. 
	\item Take your solar panel, temperature sensor, data collection system, and rod stand outside. Find a sunny location sheltered from the wind.
	\item Secure the solar panel using the tripod stand and 3-fingered clamp.
		\begin{center}
\includegraphics[width=0.4\linewidth]{setup_2_new.jpg}
\end{center}
	\item Arrange the angle of the surface of the cardboard so it is perpendicular ($90^\circ$) to the sun and the straw is pointing at the sun. Hint: When the surface of the cardboard is perpendicular to the sun, the light coming through the straw will be focused into a tight spot on the solar panel, and the shadow of the cardboard will be at its smallest size.
	\item Start a new experiment on the data collection system. 
	\item Connect the temperature sensor to the data collection system.
	\item Set up an appropriate display to view the data while it is being collected.
	\item Why did you cover the surface of the cardboard with black paper?
	\answerspace{1.5in}
	\item You will be testing three experimental situations with the cardboard positioned at: 
	\begin{itemize}
		\item $90^\circ$ relative to the light source
		\item $60^\circ$ relative to the light source
		\item $30^\circ$ relative to the light source
	\end{itemize}
	Will the position of the cardboard influence its temperature after 5 minutes at that position? Why?
	\answerspace{1.2in}
	\item Start data recording. 
	\item Record data for 5 minutes. 
	\item Write your data run number here: \rule{4cm}{0.15mm}                       .
	\item Stop data recording.
\end{enumerate}

\textit{Part 2: Test the solar panel at $60^\circ$ insolation}
\begin{enumerate}
	\item Remove the solar panel and take it to a shaded location. 
	\item Remove the straw and tape it $30^\circ$ from perpendicular on the protractor such that when the straw is perpendicular to the sun, the solar panel will be angled towards the horizon $30^\circ$, resulting in an angle of insolation of $60^\circ$. 
	\item Fan the solar panel to increase the rate of cooling. When it returns to approximately its original temperature, secure it to the tripod stand.
	\item Align the solar panel as you did before. This will angle the solar panel towards the horizon $30^\circ$ from the last setup, and it will thus receive insolation at $60^\circ$. CAUTION: Do not look directly at the sun.
	\item Start data recording. 
	\item Record data for 5 minutes. 
	\item Write your run number here: \rule{4cm}{0.15mm}                       .
	\item Stop data recording. 
	\item Remove the solar panel and take it to a shaded location. 
\end{enumerate}
\newpage
\textit{Part 3: Test the solar panel at $30^\circ$ insolation}
\begin{enumerate}
	\item Repeat the procedure in Part 2 using a $60^\circ$ tilt of the solar panel towards the horizon, and thus an angle of insolation of $30^\circ$. 
	\item Write your run number here: \rule{4cm}{0.15mm}                     .
	\item Save your file, and clean up according to your instructor's directions.
\end{enumerate}

\textbf{Data Analysis}
\begin{enumerate}
	\item Find the minimum and maximum temperatures for the three data runs and calculate the change in temperature.
	\item Enter these values in Table 1.

\begin{center}
\begin{tabular}{| M{3.0cm} | M{3.0cm} | M{3.0cm} | M{3.0cm} |} 
 \hline
 \multicolumn{4}{|c|}{Table 1: Temperature comparison at different angles of insolation} \\ \hline
 Angle of Insolation & Minimum Temperature & Maximum Temperature & Temperature Change \\ \hline
 $90^\circ$ & & & \\[15pt] \hline
 $60^\circ$ & & & \\[15pt] \hline
 $30^\circ$ & & & \\[15pt] \hline
\end{tabular}
\end{center}
	\item Plot a graph of angle of insolation on the $y$-axis with change in temperature on the $x$-axis.
	\begin{figure}[h]
{\par\centering \includegraphics[width=0.75\linewidth]{blank_graph.png} \par}
\end{figure}
\end{enumerate}
\newpage
\textbf{Analysis Questions}
\begin{enumerate}
	\item Compare your results with your predictions.
	\answerspace{1.5in}
	
	\item What is the independent variable (the parameter you controlled) in this experiment?
	\answerspace{1.0in}
	
	\item What is the dependent variable (the parameter that changed) in this experiment?
	\answerspace{1.0in}
	
	\item What parameters did you try to hold constant in this experiment (controlled variables)?
	\answerspace{1.5in}
	
	\item Is the relationship between change in temperature and angle of insolation a linear one? Explain.
	\answerspace{1.5in}
\end{enumerate}	
\newpage	
\textbf{Synthesis Questions}

Use available resources to help you answer the following questions.

\begin{enumerate}
	\item Using the results of this activity and what you know about the motion of the earth around the sun as well as the tilt of the earth's rotational axis relative to its orbital plane, explain why seasons occur. 
	\answerspace{2.0in}
	
	\item Why are seasons more pronounced the further you move away from the equator?
	\answerspace{2.0in}
\end{enumerate}	
	
\textbf{Multiple Choice Questions}

Select the best answer or completion to each of the questions or incomplete statements below.

\begin{enumerate}
	\item During wintertime in the northern hemisphere, the earth's North Pole is: 
	\begin{enumerate}[A.]
		\item Tilted towards the sun relative to the South Pole
		\item Tilted away from the sun relative to the South Pole
		\item The same distance from the sun relative to the South Pole
	\end{enumerate}
	\item During summertime in the southern hemisphere, the earth's North Pole is: 
	\begin{enumerate}[A.]
		\item Tilted towards the sun relative to the South Pole
		\item Tilted away from the sun relative to the South Pole
		\item The same distance from the sun relative to the South Pole
	\end{enumerate}
	\item At the spring and fall equinoxes, the earth's North Pole is: 
	\begin{enumerate}[A.]
		\item Tilted towards the sun relative to the South Pole
		\item Tilted away from the sun relative to the South Pole
		\item The same distance from the sun relative to the South Pole
	\end{enumerate}
	\item The warm temperatures of summer in the northern hemisphere north of the tropics occur primarily because:
	\begin{enumerate}[A.]
		\item The earth is closer to the sun
		\item The days are longer
		\item The northern hemisphere is tilted towards the sun
		\item Wind patterns change to bring warmer temperatures
	\end{enumerate}
	\item The Tropic of Cancer and Tropic of Capricorn, respectively, are circles of latitude on the earth that mark the northernmost and southernmost latitudes at which the sun may be seen directly overhead (at the June solstice and December solstice, respectively). These circles of latitude are located at approximately
	\begin{enumerate}[A.]
		\item $0^\circ$ latitude
		\item $23.5^\circ$ latitude
		\item $30.0^\circ$ latitude
		\item $60.0^\circ$ latitude
		\item $90.0^\circ$ latitude
	\end{enumerate}
\end{enumerate}

\end{document}


