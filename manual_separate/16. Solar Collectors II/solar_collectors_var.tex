\documentclass[english,twoside]{article}

\input{../labmanual_formatting_commands} %all general latex packages, commands, and definitions now here.

\begin{document}

\Lab{16}{Variables Affecting Solar Collectors}

\renewcommand{\arraystretch}{1.5}
\setlength{\arrayrulewidth}{0.4mm}
\setlength{\tabcolsep}{10pt}


\newcolumntype{M}[1]{>{\centering\arraybackslash}m{#1}}

\makelabheader %(Space for student name, etc., defined in master.tex or labmanual_formatting_commands.tex)


\textbf{Driving Question}

Light is composed of photons or bundles of radiant energy. When photons strike matter, they are absorbed, and then one of the following happens: the energy is re-emitted, or it stays within the matter in a different form. Re-emitted light can be reflected, scattered, or transmitted, depending on the direction of the re-emission. In the case of absorption, there are additional possibilities. As an example, in photovoltaic panels, the absorbed energy is converted into electricity. In a solar water heating system, the absorbed energy is converted into thermal energy (vibration and other particle motion) in the solar collector.

The ability of a solar collector to absorb photons is affected by various factors. In this experiment, you will choose one variable and test it using the KidWind Solar Thermal Exploration Kit. You will use a temperature probe to measure the change in temperature of the water as a way of measuring the absorption of light energy by the solar collector.

\textbf{Objectives}
\begin{itemize}
	\item Use a Surface Temperature Sensor to measure the change in temperature of the water in a solar collector.
	\item Create a plan to test a variable that affects solar collectors.
	\item Collect data and draw conclusions based on results.
\end{itemize}

\textbf{Materials}
\begin{itemize}
	\item Temperature sensor	
	\item KidWind Solar Thermal Exploration Kit	
	\item materials to test variable (will depend on experiment design)	
\end{itemize}
	
\textbf{Preliminary questions}
\begin{enumerate}
	\item What are the benefits of using the sun to heat water? Think about global, social, and environmental issues that can be reduced by using the sun as an energy source.
	\answerspace{3.0in}
	\item List four variables that affect how effective a solar collector can be at heating water. For each variable you list, explain how it will affect the effectiveness of the solar collector.
	\answerspace{3.5in}
	\item Which variable will you test for this experiment? Predict the results of your tests using a graph or describe the results in a paragraph.
	\answerspace{3.5in}
\end{enumerate}
\newpage
\textbf{Procedure}
\begin{enumerate}
	\item Create a plan to collect data for the variable you are testing. You will modify the solar collector 2-4 times. For example, if you are testing tubing length, you will collect data for 2-4 different lengths.
		\begin{center}
\includegraphics[width=0.5\linewidth]{experimental_setup_2.png}
\end{center}
 	\item Open \textit{SPARKvue} and build a page with one graph.
	\item Connect the temperature sensor.
	\item Set up the data-collection mode. 
	\begin{enumerate}[(a)]
		\item Click or tap Mode to open Data Collection Settings.
		\item Change the data-collection rate and duration to values that are appropriate for your plan. 
	\end{enumerate}
	\item Set up the Solar Thermal Exploration Kit to test your first modification. Note: The figure above shows a generalized setup. Your setup may differ depending on the variable you are testing.
	\item Click or tap Collect to start data collection.
	\item When data collection is complete, a graph of temperature vs. time is displayed. Click or tap the graph to examine the data. Determine the starting and maximum temperatures, as well as any other data that you need. Record these values in the data table. Note: You can also adjust the Examine line by dragging the line. 
	\item Repeat data collection until you have collected all the data you need to test your variable.
\end{enumerate}
\newpage
\textbf{Data Table}

\begin{center}
\begin{tabular}{ | M{3.0cm} | M{3.0cm} | M{3.0cm} |  M{3.0cm} |}
 \hline
 Variable (Length, Color, Material, etc) & Starting Temperature ($^\circ$C) & Maximum Temperature ($^\circ$C) & Change in Temperature ($^\circ$C) \\ \hline
 & & & \\[15pt] \hline
 & & & \\[15pt] \hline
& & & \\[15pt] \hline
& & & \\[15pt] \hline
\end{tabular}
\end{center}
 	 	 	 
\textbf{Processing the Data}
\begin{enumerate}
	\item Calculate the change in temperature for each of your modifications. 
\end{enumerate}

\textbf{Analysis Questions}
\begin{enumerate}
	\item Is there a noticeable difference in temperature change for the different modifications? If so, which system had a greater temperature change? 
	\answerspace{3.0in}
	\item Do your results match your prediction? Explain why or why not. 
	\answerspace{3.0in}
	\item Imagine someone is designing a solar collector and they have asked you for your advice. Use your results to give them advice about how to design the element of the solar collector that you tested.
	\answerspace{3.0in}
	\item If you could design an ideal solar collector, what design choices would you try to incorporate? What else could you do to improve efficiency that you did not test in this experiment?
	\answerspace{3.0in}
\end{enumerate}

\textbf{Extend}
\begin{enumerate}
	\item Share your results with the class. Then, summarize the group findings in a report. 
	\item Use the collected data to design a more efficient solar collector and test what you predict will be the best combination.
\end{enumerate}

\end{document}
