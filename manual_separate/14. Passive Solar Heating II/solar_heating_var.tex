\documentclass[english,twoside]{article}

\input{../labmanual_formatting_commands} %all general latex packages, commands, and definitions now here.

\begin{document}

\Lab{14}{Variables Affecting Passive Solar Heating}

\renewcommand{\arraystretch}{1.5}
\setlength{\arrayrulewidth}{0.4mm}
\setlength{\tabcolsep}{10pt}


\newcolumntype{M}[1]{>{\centering\arraybackslash}m{#1}}

\makelabheader %(Space for student name, etc., defined in master.tex or labmanual_formatting_commands.tex)

\textbf{Driving Question}

Buildings can be designed to retain heat in the winter while also helping them to remain cool in the summer. Insulation and heat storage are important factors in such a system. A thermal mass, a material that absorbs and stores heat, is one example of a design feature that can be used to help control the temperature in a building.

In this experiment, you will examine the effectiveness of different design features. Two model homes, one that will act as a control and one that you modify, will be heated by a light bulb. Temperature sensors will monitor their temperatures throughout a simulated day and night.

\textbf{Objectives}
\begin{itemize}
	\item Use Surface Temperature Sensors to measure temperature.
	\item Design, build, and test a model solar home.
	\item Compare your results to the data collected by other groups.
\end{itemize}

\textbf{Materials}
\begin{itemize}
	\item Temperature sensor (2)
	\item Ring stand
	\item Model home with accessories (2)	
	\item Utility clamp
	\item Solar home ``windows''
	\item Heat lamp (or 100-W lamp)	
	\item Tape
	\item Ruler
	\item Materials to test variables (will depend on experiment design)
\end{itemize}

\textbf{Preliminary questions}
\begin{enumerate}
	\item List three or four design factors, other than the presence of a thermal mass, that affect the ability for a home or building to retain heat.
	\answerspace{2.0in}
	\item If you build a building so as to increase passive solar heating, what can you do to help reflect or reduce heating in the summer when you want to keep the interior cool?
	\answerspace{2.5in}
	\item Sketch a graph of the temperature inside the solar home that is the control during one day-night heating-cooling cycle. On the same graph, sketch your prediction of the temperature inside the house that you are designing.
	\answerspace{4.0in}
\end{enumerate}

\textbf{Procedure}

\textit{Part I: Initial data collection}
\begin{enumerate}
	\item Obtain two solar homes that are matched in size and shape.
	\item Create a plan to modify the model home for the variable you are testing, and then make the necessary modifications. Tape both model homes closed.
	\item Position the two solar homes 20 cm apart, with the window sides facing each other (see figure below). Position a lamp so that it will shine down between the model solar homes. The lamp bulb should be 10 cm above the table top and equidistant from the two model homes. Do not turn on the lamp yet.
\begin{center}
\includegraphics[width=0.4\linewidth]{model_homes.png}
\end{center}	
	\item Open \textit{SPARKvue} and build a page with two graphs.
	\item Connect both temperature sensors to the data collection system.
	\item Open the data-collection settings
	\begin{enumerate}[(a)]
		\item Change Time Units to min.
		\item Change Rate to 1 sample/min and End Collection to 80 minutes.
	\end{enumerate}
	\item Measure the room temperature and record it in the data section.
	\item Position one temperature sensor (Sensor 1) in the model solar home that is the control and the other temperature sensor (Sensor 2) in the model solar home that you modified. Make sure the sensors are in the same relative location and that they are not in direct light from the lamp.

You will collect data for 80 continuous minutes. Once you have started data collection, you will turn the light on and leave it on. After 40 minutes have passed, you will turn the light off and cover the windows of the model solar homes. You will then collect ``not-lighted'' data for 40 more minutes.
	\item Click or tap Collect to start data collection. Turn on the light.
	\item After 40 minutes, turn off the light and cover the window of each model solar home with a piece of cardboard. Data collection will end after 80 minutes. 
	\item Record the maximum and final temperature values for both sensors.
	\begin{enumerate}[(a)]
		\item When data collection is complete, a graph of temperature vs. time is displayed. Click or tap the graph to examine the temperature and time values of both sensors. Note: You can also adjust the Examine line by dragging the line.
		\item Click or tap the point where the maximum temperature was recorded for Sensor 1. Record the maximum temperature value for Sensor 1 (to the nearest 0.1$^\circ$C).
		\item Now click or tap the point where the maximum temperature was recorded for Sensor 2. Record the highest temperature value for Sensor 2 (to the nearest 0.1$^\circ$C).
		\item Record the temperature at 80 minutes (round to the nearest 0.1$^\circ$C).
	\end{enumerate}
	\item Sketch or print copies of your graph as directed by your instructor. Label the two curves.
	\answerspace{4.0in}
	\item Complete Processing the Data and Analysis Questions for Part I and then continue to Part II.
\end{enumerate}

\textit{Part II: Design challenge}
\begin{enumerate}
	\item Using the information you gained in this and other experiments, design and build a model solar home that cools more slowly than the one you tested in Part I.  
	\begin{itemize}
		\item Begin with a model solar home like the one used in Part I.
		\item You may add no more than 3 cm to the thickness of the walls.
		\item You may use no more than 600 mL of thermal mass.
		\item Your home must have a window with an area of at least 150 cm$^2$.
	\end{itemize}
	\item Repeat Steps 6-11 from Part I.
	\item Complete Processing the Data and Analysis Questions for Part II.
\end{enumerate}
\newpage
\textbf{Data Table}

\textit{Part I: Initial data collection}
Room temperature ($^\circ$C): \rule{4cm}{0.15mm}
 	
\begin{center}
\begin{tabular}{ | M{4.0cm} | M{2.5cm} | M{2.5cm} | }
 \hline
 & Sensor 1 (control) & Sensor 2  (modified) \\ \hline
Maximum temperature ($^\circ$C) & & \\[15pt] \hline
 Temperature at 80 minutes ($^\circ$C) & & \\[15pt] \hline
 Temperature change ($^\circ$C) & & \\[15pt] \hline
\end{tabular}
\end{center}

\textit{Part II: Design challenge}
Room temperature ($^\circ$C): \rule{4cm}{0.15mm}

\begin{center}
\begin{tabular}{ | M{4.0cm} | M{2.5cm} | M{2.5cm} | }
 \hline
 & Sensor 1 (control) & Sensor 2  (modified) \\ \hline
Maximum temperature ($^\circ$C) & & \\[15pt] \hline
 Temperature at 80 minutes ($^\circ$C) & & \\[15pt] \hline
 Temperature change ($^\circ$C) & & \\[15pt] \hline
\end{tabular}
\end{center}
 		 	 
\textbf{Processing the Data}

\textit{Part I: Initial data collection}
\begin{enumerate}
	\item In the space provided in the data table, subtract to find the temperature changes.
	\item Share your results with the rest of the class.
\end{enumerate}

\textit{Part II: Design challenge}

	In the space provided in the data table, subtract to find the temperature changes.

\newpage
\textbf{Analysis Questions}

\textit{Part I: Initial data collection}
\begin{enumerate}
	\item Describe the modifications to your model home.
	\answerspace{3.0in}
	\item Which model solar home cooled more?
	\answerspace{2.5in}
	\item How are the ``control'' and the ``modified'' curves similar? How are they different?
	\answerspace{3.0in}
	\item How does the graph compare to the sketch you made in the preliminary questions?
	\answerspace{3.0in}
	\item Which model solar home heated more slowly?
	\answerspace{2.0in}
	\item Which model solar home cooled more slowly?
	\answerspace{2.0in}
\end{enumerate}
	
\textit{Part II: Design challenge}
\begin{enumerate}
	\item Explain why you chose the materials you did.
	\answerspace{2.5in}
	\item Compare the results with your Part I results and then compare your results with the Part II results from the other groups. Calculate $k$ for each of them. 
	\answerspace{3.0in}
\end{enumerate}

\textbf{Extend}
\begin{enumerate}
	\item Run the experiment for two or more consecutive ``daily'' cycles of four hours or longer.
	\item Design an experiment to test other types of thermal mass, such as stones or phase-change materials.
	\item Design an experiment to test other variables affecting a solar home, such as color, window material, window size, and insulation type.
	\item What design factors about your school allow it to take advantage of passive solar heating? Are there things you could do at your school to improve passive heating as well as to reduce the need for using electricity to cool the building in the summer?
\end{enumerate}

\end{document}
